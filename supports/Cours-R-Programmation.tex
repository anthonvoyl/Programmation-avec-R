% Options for packages loaded elsewhere
\PassOptionsToPackage{unicode}{hyperref}
\PassOptionsToPackage{hyphens}{url}
\PassOptionsToPackage{dvipsnames,svgnames,x11names}{xcolor}
%
\documentclass[
  ignorenonframetext,
]{beamer}
\usepackage{pgfpages}
\setbeamertemplate{caption}[numbered]
\setbeamertemplate{caption label separator}{: }
\setbeamercolor{caption name}{fg=normal text.fg}
\beamertemplatenavigationsymbolsempty
% Prevent slide breaks in the middle of a paragraph
\widowpenalties 1 10000
\raggedbottom
\setbeamertemplate{part page}{
  \centering
  \begin{beamercolorbox}[sep=16pt,center]{part title}
    \usebeamerfont{part title}\insertpart\par
  \end{beamercolorbox}
}
\setbeamertemplate{section page}{
  \centering
  \begin{beamercolorbox}[sep=12pt,center]{part title}
    \usebeamerfont{section title}\insertsection\par
  \end{beamercolorbox}
}
\setbeamertemplate{subsection page}{
  \centering
  \begin{beamercolorbox}[sep=8pt,center]{part title}
    \usebeamerfont{subsection title}\insertsubsection\par
  \end{beamercolorbox}
}
\AtBeginPart{
  \frame{\partpage}
}
\AtBeginSection{
  \ifbibliography
  \else
    \frame{\sectionpage}
  \fi
}
\AtBeginSubsection{
  \frame{\subsectionpage}
}
\usepackage{amsmath,amssymb}
\usepackage{iftex}
\ifPDFTeX
  \usepackage[T1]{fontenc}
  \usepackage[utf8]{inputenc}
  \usepackage{textcomp} % provide euro and other symbols
\else % if luatex or xetex
  \usepackage{unicode-math} % this also loads fontspec
  \defaultfontfeatures{Scale=MatchLowercase}
  \defaultfontfeatures[\rmfamily]{Ligatures=TeX,Scale=1}
\fi
\usepackage{lmodern}
\usetheme[]{AnnArbor}
\usecolortheme{dolphin}
\usefonttheme{structurebold}
\ifPDFTeX\else
  % xetex/luatex font selection
\fi
% Use upquote if available, for straight quotes in verbatim environments
\IfFileExists{upquote.sty}{\usepackage{upquote}}{}
\IfFileExists{microtype.sty}{% use microtype if available
  \usepackage[]{microtype}
  \UseMicrotypeSet[protrusion]{basicmath} % disable protrusion for tt fonts
}{}
\makeatletter
\@ifundefined{KOMAClassName}{% if non-KOMA class
  \IfFileExists{parskip.sty}{%
    \usepackage{parskip}
  }{% else
    \setlength{\parindent}{0pt}
    \setlength{\parskip}{6pt plus 2pt minus 1pt}}
}{% if KOMA class
  \KOMAoptions{parskip=half}}
\makeatother
\usepackage{xcolor}
\newif\ifbibliography
\usepackage{color}
\usepackage{fancyvrb}
\newcommand{\VerbBar}{|}
\newcommand{\VERB}{\Verb[commandchars=\\\{\}]}
\DefineVerbatimEnvironment{Highlighting}{Verbatim}{commandchars=\\\{\}}
% Add ',fontsize=\small' for more characters per line
\usepackage{framed}
\definecolor{shadecolor}{RGB}{248,248,248}
\newenvironment{Shaded}{\begin{snugshade}}{\end{snugshade}}
\newcommand{\AlertTok}[1]{\textcolor[rgb]{0.94,0.16,0.16}{#1}}
\newcommand{\AnnotationTok}[1]{\textcolor[rgb]{0.56,0.35,0.01}{\textbf{\textit{#1}}}}
\newcommand{\AttributeTok}[1]{\textcolor[rgb]{0.13,0.29,0.53}{#1}}
\newcommand{\BaseNTok}[1]{\textcolor[rgb]{0.00,0.00,0.81}{#1}}
\newcommand{\BuiltInTok}[1]{#1}
\newcommand{\CharTok}[1]{\textcolor[rgb]{0.31,0.60,0.02}{#1}}
\newcommand{\CommentTok}[1]{\textcolor[rgb]{0.56,0.35,0.01}{\textit{#1}}}
\newcommand{\CommentVarTok}[1]{\textcolor[rgb]{0.56,0.35,0.01}{\textbf{\textit{#1}}}}
\newcommand{\ConstantTok}[1]{\textcolor[rgb]{0.56,0.35,0.01}{#1}}
\newcommand{\ControlFlowTok}[1]{\textcolor[rgb]{0.13,0.29,0.53}{\textbf{#1}}}
\newcommand{\DataTypeTok}[1]{\textcolor[rgb]{0.13,0.29,0.53}{#1}}
\newcommand{\DecValTok}[1]{\textcolor[rgb]{0.00,0.00,0.81}{#1}}
\newcommand{\DocumentationTok}[1]{\textcolor[rgb]{0.56,0.35,0.01}{\textbf{\textit{#1}}}}
\newcommand{\ErrorTok}[1]{\textcolor[rgb]{0.64,0.00,0.00}{\textbf{#1}}}
\newcommand{\ExtensionTok}[1]{#1}
\newcommand{\FloatTok}[1]{\textcolor[rgb]{0.00,0.00,0.81}{#1}}
\newcommand{\FunctionTok}[1]{\textcolor[rgb]{0.13,0.29,0.53}{\textbf{#1}}}
\newcommand{\ImportTok}[1]{#1}
\newcommand{\InformationTok}[1]{\textcolor[rgb]{0.56,0.35,0.01}{\textbf{\textit{#1}}}}
\newcommand{\KeywordTok}[1]{\textcolor[rgb]{0.13,0.29,0.53}{\textbf{#1}}}
\newcommand{\NormalTok}[1]{#1}
\newcommand{\OperatorTok}[1]{\textcolor[rgb]{0.81,0.36,0.00}{\textbf{#1}}}
\newcommand{\OtherTok}[1]{\textcolor[rgb]{0.56,0.35,0.01}{#1}}
\newcommand{\PreprocessorTok}[1]{\textcolor[rgb]{0.56,0.35,0.01}{\textit{#1}}}
\newcommand{\RegionMarkerTok}[1]{#1}
\newcommand{\SpecialCharTok}[1]{\textcolor[rgb]{0.81,0.36,0.00}{\textbf{#1}}}
\newcommand{\SpecialStringTok}[1]{\textcolor[rgb]{0.31,0.60,0.02}{#1}}
\newcommand{\StringTok}[1]{\textcolor[rgb]{0.31,0.60,0.02}{#1}}
\newcommand{\VariableTok}[1]{\textcolor[rgb]{0.00,0.00,0.00}{#1}}
\newcommand{\VerbatimStringTok}[1]{\textcolor[rgb]{0.31,0.60,0.02}{#1}}
\newcommand{\WarningTok}[1]{\textcolor[rgb]{0.56,0.35,0.01}{\textbf{\textit{#1}}}}
\setlength{\emergencystretch}{3em} % prevent overfull lines
\providecommand{\tightlist}{%
  \setlength{\itemsep}{0pt}\setlength{\parskip}{0pt}}
\setcounter{secnumdepth}{5}
\titlegraphic{\centering \includegraphics[width=8cm]{igpde.jpeg}}
\ifLuaTeX
  \usepackage{selnolig}  % disable illegal ligatures
\fi
\IfFileExists{bookmark.sty}{\usepackage{bookmark}}{\usepackage{hyperref}}
\IfFileExists{xurl.sty}{\usepackage{xurl}}{} % add URL line breaks if available
\urlstyle{same}
\hypersetup{
  pdftitle={Programmation en R},
  pdfauthor={Alexis Gabadinho},
  colorlinks=true,
  linkcolor={Maroon},
  filecolor={Maroon},
  citecolor={Blue},
  urlcolor={blue},
  pdfcreator={LaTeX via pandoc}}

\title{Programmation en R}
\author{Alexis Gabadinho}
\date{2023-12-18}

\begin{document}
\frame{\titlepage}

\begin{frame}[allowframebreaks]
  \tableofcontents[hideallsubsections]
\end{frame}
\hypertarget{introduction}{%
\section{Introduction}\label{introduction}}

\begin{frame}{Objectifs pédagogiques}
\protect\hypertarget{objectifs-puxe9dagogiques}{}
\begin{itemize}
\tightlist
\item
  Définir les fondamentaux de R
\item
  Créer une fonction
\item
  Gérer les attributs d'un objet
\item
  Gérer les paramètres d'une fonction
\end{itemize}
\end{frame}

\begin{frame}{Supports}
\protect\hypertarget{supports}{}
\end{frame}

\begin{frame}[fragile]{Prérequis}
\protect\hypertarget{pruxe9requis}{}
\begin{itemize}
\tightlist
\item
  Savoir travailler en R : RStudio ou Rgui
\item
  Savoir installer, charger un package
\item
  Savoir naviguer dans la documentation des packages
\item
  Savoir visualiser la structure d'une table
\item
  Connaître le sens de l'assignation
\item
  Connaître la fonction data.frame
\item
  Avec le package \texttt{dplyr} : savoir créer une colonne, faire des
  statistiques
\item
  Savoir utiliser le pipe \texttt{\%\textgreater{}\%}
\item
  Savoir faire un graphique simple (courbe, graphique en barre)
\end{itemize}
\end{frame}

\hypertarget{les-principes-du-langage-r}{%
\section{Les principes du langage R}\label{les-principes-du-langage-r}}

\begin{frame}[fragile]{Objets et classes}
\protect\hypertarget{objets-et-classes}{}
\begin{itemize}
\tightlist
\item
  En R tout est \textbf{objet}
\item
  Chaque objet a une (ou plusieurs) classes
\item
  La fonction \texttt{class()} renvoie la classe d'un objet
\end{itemize}

\tiny

\begin{Shaded}
\begin{Highlighting}[]
\NormalTok{a }\OtherTok{\textless{}{-}} \DecValTok{1}
\FunctionTok{class}\NormalTok{(a)}
\end{Highlighting}
\end{Shaded}

\begin{verbatim}
## [1] "numeric"
\end{verbatim}

\normalsize

\begin{itemize}
\item
  Ce sont les fonctions génériques (voir plus loin) qui permettent
  ensuite d'associer un comportement à une classe
\item
  La classe d'un objet est \textbf{modifiable} ´
\end{itemize}

\tiny

\begin{Shaded}
\begin{Highlighting}[]
\FunctionTok{class}\NormalTok{(a) }\OtherTok{\textless{}{-}} \FunctionTok{c}\NormalTok{(}\StringTok{"truc"}\NormalTok{,}\StringTok{"machin"}\NormalTok{)}
\FunctionTok{class}\NormalTok{(a)}
\end{Highlighting}
\end{Shaded}

\begin{verbatim}
## [1] "truc"   "machin"
\end{verbatim}

\normalsize
\end{frame}

\begin{frame}[fragile]{Stockage des objets}
\protect\hypertarget{stockage-des-objets}{}
\begin{itemize}
\tightlist
\item
  La fonction \texttt{storage.mode()} renvoie la manière dont sont
  représentées les données en mémoire (pour une donnée, il n'y a qu'un
  mode de stockage)
\end{itemize}

\tiny

\begin{Shaded}
\begin{Highlighting}[]
\NormalTok{d }\OtherTok{\textless{}{-}} \FunctionTok{Sys.Date}\NormalTok{()}
\NormalTok{d}
\end{Highlighting}
\end{Shaded}

\begin{verbatim}
## [1] "2023-12-18"
\end{verbatim}

\normalsize

\begin{itemize}
\tightlist
\item
  Une date est stockée sous la forme d'un nombre décimal
\end{itemize}

\tiny

\begin{Shaded}
\begin{Highlighting}[]
\FunctionTok{storage.mode}\NormalTok{(d)}
\end{Highlighting}
\end{Shaded}

\begin{verbatim}
## [1] "double"
\end{verbatim}

\normalsize

\begin{itemize}
\tightlist
\item
  Pour les types de base, classe et stockage sont
\end{itemize}

\tiny

\begin{Shaded}
\begin{Highlighting}[]
\NormalTok{i }\OtherTok{\textless{}{-}} \DecValTok{12}\DataTypeTok{L}
\FunctionTok{storage.mode}\NormalTok{(i)}
\end{Highlighting}
\end{Shaded}

\begin{verbatim}
## [1] "integer"
\end{verbatim}

\normalsize
\end{frame}

\begin{frame}[fragile]{Constantes et symboles (1)}
\protect\hypertarget{constantes-et-symboles-1}{}
\begin{itemize}
\tightlist
\item
  Commençant par des chiffres, ou un point suivi de chiffres : un nombre
  (\texttt{numeric}, \texttt{double})
\end{itemize}

\tiny

\begin{Shaded}
\begin{Highlighting}[]
\DecValTok{12}
\end{Highlighting}
\end{Shaded}

\begin{verbatim}
## [1] 12
\end{verbatim}

\normalsize

\begin{itemize}
\tightlist
\item
  Commençant par des quotes simples (') ou doubles (``) : une chaîne de
  caractères
\end{itemize}

\tiny

\begin{Shaded}
\begin{Highlighting}[]
\StringTok{\textquotesingle{}abcd\textquotesingle{}}
\end{Highlighting}
\end{Shaded}

\begin{verbatim}
## [1] "abcd"
\end{verbatim}

\normalsize
\end{frame}

\begin{frame}[fragile]{Constantes et symboles (2)}
\protect\hypertarget{constantes-et-symboles-2}{}
\begin{itemize}
\tightlist
\item
  Pas de quotes, commençant par une lettre ou un point, suivi de
  lettres, chiffres, points, ou blanc soulignés : le nom de quelque
  chose, un « symbole » dont la signification peut varier
\end{itemize}

\tiny

\begin{Shaded}
\begin{Highlighting}[]
\NormalTok{sum}
\end{Highlighting}
\end{Shaded}

\begin{verbatim}
## function (..., na.rm = FALSE)  .Primitive("sum")
\end{verbatim}

\normalsize

\begin{itemize}
\tightlist
\item
  Séquences de caractères prédéfinies, \textbf{mots-clés} du langage
  dont la signification est inaltérable
\end{itemize}

\tiny

\begin{Shaded}
\begin{Highlighting}[]
\ConstantTok{TRUE}
\end{Highlighting}
\end{Shaded}

\begin{verbatim}
## [1] TRUE
\end{verbatim}

\normalsize
\end{frame}

\begin{frame}[fragile]{Fonctions et opérateurs (1)}
\protect\hypertarget{fonctions-et-opuxe9rateurs-1}{}
\begin{itemize}
\tightlist
\item
  Il y a deux façons d'obtenir un résultat :

  \begin{itemize}
  \tightlist
  \item
    Par des appels de \textbf{fonction} : un nom de fonction (un «
    symbole ») et, entre parenthèses, séparés par des virgules, les
    paramètres de la fonction passés soit par \textbf{position} soit par
    \textbf{nom}
  \end{itemize}
\end{itemize}

\tiny

\begin{Shaded}
\begin{Highlighting}[]
\FunctionTok{library}\NormalTok{(rio)}
\NormalTok{fichier }\OtherTok{\textless{}{-}} \StringTok{"/home/alex/Devel/Cours{-}R{-}Programmation/data/dpt2022.csv"}
\FunctionTok{import}\NormalTok{(fichier, }\AttributeTok{as.is=}\ConstantTok{TRUE}\NormalTok{)}
\end{Highlighting}
\end{Shaded}

\begin{verbatim}
## # A tibble: 3,835,767 x 5
##     sexe preusuel       annais dpt   nombre
##    <int> <chr>          <chr>  <chr>  <int>
##  1     1 _PRENOMS_RARES 1900   02         7
##  2     1 _PRENOMS_RARES 1900   04         9
##  3     1 _PRENOMS_RARES 1900   05         8
##  4     1 _PRENOMS_RARES 1900   06        23
##  5     1 _PRENOMS_RARES 1900   07         9
##  6     1 _PRENOMS_RARES 1900   08         4
##  7     1 _PRENOMS_RARES 1900   09         6
##  8     1 _PRENOMS_RARES 1900   10         3
##  9     1 _PRENOMS_RARES 1900   11        11
## 10     1 _PRENOMS_RARES 1900   12         7
## # i 3,835,757 more rows
\end{verbatim}

\normalsize
\end{frame}

\begin{frame}[fragile]{Fonctions et opérateurs (2)}
\protect\hypertarget{fonctions-et-opuxe9rateurs-2}{}
\begin{itemize}
\tightlist
\item
  Par des appels à des \textbf{opérateurs} : une expression, l'opérateur
  et une autre expression
\end{itemize}

\tiny

\begin{Shaded}
\begin{Highlighting}[]
\DecValTok{1}\SpecialCharTok{+}\DecValTok{1}
\end{Highlighting}
\end{Shaded}

\begin{verbatim}
## [1] 2
\end{verbatim}

\begin{Shaded}
\begin{Highlighting}[]
\NormalTok{a }\OtherTok{\textless{}{-}} \DecValTok{1}
\end{Highlighting}
\end{Shaded}

\normalsize

\begin{itemize}
\tightlist
\item
  Certains opérateurs sont fournis par des librairies optionnelles
\end{itemize}

\tiny

\begin{Shaded}
\begin{Highlighting}[]
\FunctionTok{library}\NormalTok{(tidyr)}
\NormalTok{fichier }\SpecialCharTok{\%\textgreater{}\%} \FunctionTok{import}\NormalTok{(}\AttributeTok{as.is=}\ConstantTok{TRUE}\NormalTok{)}
\end{Highlighting}
\end{Shaded}

\begin{verbatim}
## # A tibble: 3,835,767 x 5
##     sexe preusuel       annais dpt   nombre
##    <int> <chr>          <chr>  <chr>  <int>
##  1     1 _PRENOMS_RARES 1900   02         7
##  2     1 _PRENOMS_RARES 1900   04         9
##  3     1 _PRENOMS_RARES 1900   05         8
##  4     1 _PRENOMS_RARES 1900   06        23
##  5     1 _PRENOMS_RARES 1900   07         9
##  6     1 _PRENOMS_RARES 1900   08         4
##  7     1 _PRENOMS_RARES 1900   09         6
##  8     1 _PRENOMS_RARES 1900   10         3
##  9     1 _PRENOMS_RARES 1900   11        11
## 10     1 _PRENOMS_RARES 1900   12         7
## # i 3,835,757 more rows
\end{verbatim}

\normalsize

\begin{itemize}
\tightlist
\item
  Tout appel produit un résultat (pas nécessairement visible, comme par
  exemple avec l'opérateur \texttt{\textless{}-})
\end{itemize}
\end{frame}

\begin{frame}[fragile]{La boucle de l'invite de commande}
\protect\hypertarget{la-boucle-de-linvite-de-commande}{}
\begin{itemize}
\tightlist
\item
  Lecture d'une chaîne de caractères: \texttt{scan()}
\item
  Interprétation du code à exécuter (syntaxe): \texttt{parse()}
\item
  Evaluation (produit un résultat en mémoire): \texttt{eval()}
\item
  Impression (affichage du résultat dans la console): \texttt{print()}
\end{itemize}
\end{frame}

\begin{frame}[fragile]{Les fonctions \texttt{parse()} et
\texttt{eval()}}
\protect\hypertarget{les-fonctions-parse-et-eval}{}
\begin{itemize}
\tightlist
\item
  Le rôle de la fonction \texttt{parse()} est de faire les contrôles de
  syntaxe. Au passage, elle transforme une séquence de caractères en une
  structure interne, une « expression » ordonnançant les futurs calculs
\end{itemize}

\tiny

\begin{Shaded}
\begin{Highlighting}[]
\NormalTok{e }\OtherTok{\textless{}{-}} \FunctionTok{parse}\NormalTok{(}\AttributeTok{text=}\StringTok{"6*7"}\NormalTok{)}
\NormalTok{e}
\end{Highlighting}
\end{Shaded}

\begin{verbatim}
## expression(6 * 7)
\end{verbatim}

\normalsize - L'expression contient l'arborescence des calculs à
effectuer

\tiny

\begin{Shaded}
\begin{Highlighting}[]
\FunctionTok{as.list}\NormalTok{(e[[}\DecValTok{1}\NormalTok{]])}
\end{Highlighting}
\end{Shaded}

\begin{verbatim}
## [[1]]
## `*`
## 
## [[2]]
## [1] 6
## 
## [[3]]
## [1] 7
\end{verbatim}

\normalsize

\begin{itemize}
\tightlist
\item
  La fonction \texttt{eval()} appliquée à une \textbf{expression},
  permet d'obtenir un résultat. Elle est le coeur de R.
\end{itemize}

\tiny

\begin{Shaded}
\begin{Highlighting}[]
\FunctionTok{eval}\NormalTok{(e)}
\end{Highlighting}
\end{Shaded}

\begin{verbatim}
## [1] 42
\end{verbatim}

\normalsize
\end{frame}

\begin{frame}[fragile]{La fonction \texttt{quote()}}
\protect\hypertarget{la-fonction-quote}{}
\begin{itemize}
\tightlist
\item
  Comme une majorité de fonctions, la fonction \texttt{eval()} commence
  par prendre la valeur de son argument puis travaille sur cette valeur.
  Au final, comme la fonctionnalité est d'évaluer, l'argument est évalué
  deux fois.
\item
  La fonction \texttt{quote()} fait, elle, partie des fonctions qui ne
  commencent pas par prendre la valeur de leur argument. La fonction
  quote restitue son argument sans tenter de l'évaluer.
\end{itemize}

\tiny

\begin{Shaded}
\begin{Highlighting}[]
\NormalTok{e }\OtherTok{\textless{}{-}} \FunctionTok{quote}\NormalTok{(}\DecValTok{6}\SpecialCharTok{*}\DecValTok{7}\NormalTok{)}
\FunctionTok{class}\NormalTok{(e)}
\end{Highlighting}
\end{Shaded}

\begin{verbatim}
## [1] "call"
\end{verbatim}

\normalsize

\tiny

\begin{Shaded}
\begin{Highlighting}[]
\NormalTok{e}
\end{Highlighting}
\end{Shaded}

\begin{verbatim}
## 6 * 7
\end{verbatim}

\normalsize

\tiny

\begin{Shaded}
\begin{Highlighting}[]
\FunctionTok{eval}\NormalTok{(e)}
\end{Highlighting}
\end{Shaded}

\begin{verbatim}
## [1] 42
\end{verbatim}

\normalsize

\begin{itemize}
\tightlist
\item
  Il existe de nombreuses fonctions qui travaillent ainsi sur des objets
  de nature « expression » autorisant ainsi la construction de
  programmes constructeurs de programmes sans passer par une
  représentation sous forme de chaîne de caractères.
\end{itemize}
\end{frame}

\begin{frame}{La fonction \texttt{print()}}
\protect\hypertarget{la-fonction-print}{}
\begin{itemize}
\tightlist
\item
  Toute opération en R produit un résultat, c'est à dire un objet stocké
  quelque part en mémoire.
\item
  La dernière étape de la boucle est donc la visualisation de ce
  résultat.
\item
  Celui ci peut être simple (une chaîne de caractères) : juste entouré
  de quotes ou \ldots{}
\item
  \ldots{} un peu plus compliqué parce que nécessitant des choix de
  présentation (un nombre, un data.frame) ou être une structure encore
  plus complexe (un tableau, un graphique\ldots)
\end{itemize}
\end{frame}

\begin{frame}[fragile]{La fonction \texttt{print()}: Exemple}
\protect\hypertarget{la-fonction-print-exemple}{}
\tiny

\begin{Shaded}
\begin{Highlighting}[]
\FunctionTok{library}\NormalTok{(ggformula)}
\end{Highlighting}
\end{Shaded}

\begin{verbatim}
## Le chargement a nécessité le package : ggplot2
\end{verbatim}

\begin{verbatim}
## Le chargement a nécessité le package : scales
\end{verbatim}

\begin{verbatim}
## Le chargement a nécessité le package : ggridges
\end{verbatim}

\begin{verbatim}
## 
## New to ggformula?  Try the tutorials: 
##  learnr::run_tutorial("introduction", package = "ggformula")
##  learnr::run_tutorial("refining", package = "ggformula")
\end{verbatim}

\begin{Shaded}
\begin{Highlighting}[]
\NormalTok{g }\OtherTok{\textless{}{-}}\NormalTok{ mtcars }\SpecialCharTok{\%\textgreater{}\%} \FunctionTok{gf\_bar}\NormalTok{( }\SpecialCharTok{\textasciitilde{}}\NormalTok{ cyl, }\AttributeTok{fill=}\StringTok{"blue"}\NormalTok{) }\SpecialCharTok{+} \FunctionTok{theme\_minimal}\NormalTok{()}
\NormalTok{g}
\end{Highlighting}
\end{Shaded}

\begin{center}\includegraphics[width=0.7\linewidth]{Cours-R-Programmation_files/figure-beamer/unnamed-chunk-19-1} \end{center}

\normalsize
\end{frame}

\begin{frame}[fragile]{La fonction \texttt{invisible()}}
\protect\hypertarget{la-fonction-invisible}{}
\begin{itemize}
\tightlist
\item
  Toute fonction produit un résultat, tout opérateur est une fonction,
  \texttt{\textless{}-} est un opérateur.
\item
  Pourtant l'affectation ne montre rien
\end{itemize}

\tiny

\begin{Shaded}
\begin{Highlighting}[]
\NormalTok{x }\OtherTok{\textless{}{-}} \DecValTok{6} \SpecialCharTok{*} \DecValTok{7}
\end{Highlighting}
\end{Shaded}

\normalsize - L'utilisation des parenthèses permet de montrer le
résultat du calcul

\tiny

\begin{Shaded}
\begin{Highlighting}[]
\NormalTok{(x }\OtherTok{\textless{}{-}} \DecValTok{6}\SpecialCharTok{*}\DecValTok{7}\NormalTok{)}
\end{Highlighting}
\end{Shaded}

\begin{verbatim}
## [1] 42
\end{verbatim}

\normalsize

\begin{itemize}
\tightlist
\item
  Il est possible de définir des fonctions avec un résultat marqué comme
  invisible et qui, comme tel, ne sera pas affiché par print.
\end{itemize}

\tiny

\begin{Shaded}
\begin{Highlighting}[]
\DecValTok{6}\SpecialCharTok{*}\FunctionTok{print}\NormalTok{(}\DecValTok{7}\NormalTok{)}
\end{Highlighting}
\end{Shaded}

\begin{verbatim}
## [1] 7
\end{verbatim}

\begin{verbatim}
## [1] 42
\end{verbatim}

\normalsize - La fonction \texttt{invisible()} marque l'évaluation de
son argument comme ne devant pas être affiché.

\tiny

\begin{Shaded}
\begin{Highlighting}[]
\FunctionTok{invisible}\NormalTok{(}\DecValTok{6}\SpecialCharTok{*}\FunctionTok{print}\NormalTok{(}\DecValTok{7}\NormalTok{))}
\end{Highlighting}
\end{Shaded}

\begin{verbatim}
## [1] 7
\end{verbatim}

\normalsize
\end{frame}

\begin{frame}[fragile]{Symboles et objets (1)}
\protect\hypertarget{symboles-et-objets-1}{}
\begin{itemize}
\tightlist
\item
  De façon quasi systématique, l'appel à une fonction R crée quelque
  chose quelque part en mémoire.
\item
  La quantité de mémoire occupée n'est pas définie a priori, c'est la
  fonction qui se charge de ``prendre'' ce dont elle a besoin.
\item
  Le résultat de l'appel d'une fonction devrait donc être une indication
  de l'endroit où la fonction a créé quelque chose, mais cela ne serait
  guère pratique de manipuler des adresses mémoire.
\item
  A la place, on utilise l'assignation \texttt{\textless{}-} pour
  associer un \textbf{nom} à l'\textbf{objet} créé (ce qui permet de le
  réutiliser)
\end{itemize}

\tiny

\begin{Shaded}
\begin{Highlighting}[]
\NormalTok{prenoms2022 }\OtherTok{\textless{}{-}} \FunctionTok{import}\NormalTok{(}\StringTok{"../data/dpt2022.csv"}\NormalTok{, }\AttributeTok{as.is=}\ConstantTok{TRUE}\NormalTok{)}
\end{Highlighting}
\end{Shaded}

\normalsize
\end{frame}

\begin{frame}[fragile]{Symboles et objets (2)}
\protect\hypertarget{symboles-et-objets-2}{}
\begin{itemize}
\tightlist
\item
  Contrairement à la logique de beaucoup de langages de programmation,
  des noms comme \texttt{prenoms2022}, \texttt{import} ne correspondent
  pas à des cases contenant des choses (une table, une fonction) qu'il
  aurait fallu pré-déclarer.
\item
  Ce sont plutôt des étiquettes (des symboles), qu'on appose a
  posteriori à coté des objets pour pouvoir en parler
\item
  L'instruction suivante ne fait aucune copie, mais créé un second nom
  pour identifier le même objet
\end{itemize}

\tiny

\begin{Shaded}
\begin{Highlighting}[]
\NormalTok{prenoms2022b }\OtherTok{\textless{}{-}}\NormalTok{ prenoms2022}
\end{Highlighting}
\end{Shaded}

\normalsize
\end{frame}

\begin{frame}[fragile]{Association d'un nom}
\protect\hypertarget{association-dun-nom}{}
\begin{enumerate}
\tightlist
\item
  Chargement des données en mémoire
\end{enumerate}

\tiny

\begin{Shaded}
\begin{Highlighting}[]
\NormalTok{prenoms2022 }\OtherTok{\textless{}{-}} \FunctionTok{import}\NormalTok{(}\StringTok{"../data/dpt2022.csv"}\NormalTok{, }\AttributeTok{as.is=}\ConstantTok{TRUE}\NormalTok{)}
\end{Highlighting}
\end{Shaded}

\normalsize

\begin{enumerate}
\setcounter{enumi}{1}
\tightlist
\item
  Enregistrement d'un nom, l'objet apparaît dans l'environnement
  \texttt{.GlobalEnv}
\end{enumerate}

\tiny

\begin{Shaded}
\begin{Highlighting}[]
\FunctionTok{ls}\NormalTok{()}
\end{Highlighting}
\end{Shaded}

\begin{verbatim}
##  [1] "a"              "d"              "def.chunk.hook" "e"             
##  [5] "fichier"        "g"              "i"              "prenoms2022"   
##  [9] "prenoms2022b"   "x"
\end{verbatim}

\normalsize

\begin{enumerate}
\setcounter{enumi}{2}
\tightlist
\item
  Association du nom aux données
\end{enumerate}
\end{frame}

\begin{frame}[fragile]{Symboles ou chaînes de caractères (1)}
\protect\hypertarget{symboles-ou-chauxeenes-de-caractuxe8res-1}{}
\begin{itemize}
\tightlist
\item
  Les symboles ne sont pas que des étiquettes apposées sur des objets,
  ils peuvent aussi être utilisés comme arguments sans faire référence à
  l'objet qu'ils pourraient désigner (il pourraient n'en désigner
  aucun!). Et des ponts existent vers le type ``character''.
\item
  Ici on convertit le symbole \texttt{x} PAS sa valeur
\end{itemize}

\tiny

\begin{Shaded}
\begin{Highlighting}[]
\FunctionTok{as.character}\NormalTok{(}\FunctionTok{quote}\NormalTok{(x))}
\end{Highlighting}
\end{Shaded}

\begin{verbatim}
## [1] "x"
\end{verbatim}

\normalsize

\begin{itemize}
\tightlist
\item
  Ici on fabrique un symbole à partir d'une chaîne de caractère
\end{itemize}

\tiny

\begin{Shaded}
\begin{Highlighting}[]
\FunctionTok{as.name}\NormalTok{(}\StringTok{"x"}\NormalTok{)}
\end{Highlighting}
\end{Shaded}

\begin{verbatim}
## x
\end{verbatim}

\normalsize
\end{frame}

\begin{frame}[fragile]{Symboles ou chaînes de caractères (2)}
\protect\hypertarget{symboles-ou-chauxeenes-de-caractuxe8res-2}{}
\begin{itemize}
\tightlist
\item
  On peut aussi explicitement naviguer dans le lien entre un symbole,
  exprimé comme chaîne de caractères, et l'objet associé.
\end{itemize}

\tiny

\begin{Shaded}
\begin{Highlighting}[]
\NormalTok{x }\OtherTok{\textless{}{-}} \DecValTok{42}
\FunctionTok{get}\NormalTok{(}\StringTok{"x"}\NormalTok{)}
\end{Highlighting}
\end{Shaded}

\begin{verbatim}
## [1] 42
\end{verbatim}

\normalsize - Par contre l'instruction suivante renvoie une erreur car
cela revient à faire \texttt{get(42)}

\tiny

\begin{Shaded}
\begin{Highlighting}[]
\FunctionTok{get}\NormalTok{(x)}
\end{Highlighting}
\end{Shaded}

\normalsize

\begin{itemize}
\tightlist
\item
  Assignation d'une valeur à un nom dans un environement
\end{itemize}

\tiny

\begin{Shaded}
\begin{Highlighting}[]
\FunctionTok{assign}\NormalTok{(}\StringTok{"x"}\NormalTok{,pi)}
\end{Highlighting}
\end{Shaded}

\normalsize
\end{frame}

\begin{frame}{Les objets ne sont pas modifiables (1)}
\protect\hypertarget{les-objets-ne-sont-pas-modifiables-1}{}
\begin{itemize}
\tightlist
\item
  En règle quasi-générale (exceptions : data.table, R6), il est
  impossible de modifier un objet de R.
\item
  Fonctionnellement, si on veut faire une modification à un objet, par
  exemple une table de données, associée à un symbole donné :

  \begin{itemize}
  \tightlist
  \item
    on construit une copie modifiée de la table
  \item
    on associe la copie modifiée au symbole
  \item
    la précédente table associée au symbole est alors perdue parce que
    plus référençable (sauf si on l'avait associée à un deuxième
    symbole).
  \end{itemize}
\item
  En pratique, c'est R qui se charge de minimiser le nombre de réelles
  copies et de supprimer de la mémoire les objets ``perdus'' (non
  référençables).
\end{itemize}
\end{frame}

\begin{frame}[fragile]{Les objets ne sont pas modifiables (2)}
\protect\hypertarget{les-objets-ne-sont-pas-modifiables-2}{}
\begin{itemize}
\tightlist
\item
  Par exemple, avec le package dplyr, pour convertir la colonne
  \texttt{sexe} de la table `nanopop' en numérique dans une nouvelle
  variable sexe2, on écrira :
\end{itemize}

\tiny

\begin{Shaded}
\begin{Highlighting}[]
\NormalTok{nanopop }\SpecialCharTok{\%\textgreater{}\%} \FunctionTok{mutate}\NormalTok{(}\AttributeTok{sexe2=}\FunctionTok{as.numeric}\NormalTok{(sexe))}
\end{Highlighting}
\end{Shaded}

\normalsize

\begin{itemize}
\tightlist
\item
  En sortie il y a en mémoire :

  \begin{itemize}
  \tightlist
  \item
    le résultat, une table avec la colonne supplémentaire , qui est
    juste affiché,
  \item
    mais aussi la table originale, qui est toujours associée au symbole
    nanopop.
  \end{itemize}
\item
  Pour ``modifier'' l'objet, il faut écrire explicitement
\end{itemize}

\tiny

\begin{Shaded}
\begin{Highlighting}[]
\NormalTok{nanopop }\SpecialCharTok{\%\textgreater{}\%} \FunctionTok{mutate}\NormalTok{(}\AttributeTok{sexe2=}\FunctionTok{as.numeric}\NormalTok{(sexe)) }\OtherTok{{-}\textgreater{}}\NormalTok{ nanopop}
\end{Highlighting}
\end{Shaded}

\normalsize
\end{frame}

\hypertarget{les-bases-du-langage}{%
\section{Les bases du langage}\label{les-bases-du-langage}}

\begin{frame}[fragile]{Les type élémentaires (atomiques)}
\protect\hypertarget{les-type-uxe9luxe9mentaires-atomiques}{}
\begin{itemize}
\tightlist
\item
  En R, la structure de données de base est le vecteur : ensemble de
  données qui sont toutes du même ``type'' de base (aussi dit
  ``atomique'')
\item
  Les types de base en R (\texttt{atomic}):

  \begin{itemize}
  \tightlist
  \item
    logical
  \item
    integer
  \item
    numeric (double)
  \item
    complex
  \item
    character
  \item
    raw
  \end{itemize}
\item
  Les données dans ces types de base ne peuvent exister en dehors d'une
  structure de vecteur (principe 3).
\end{itemize}
\end{frame}

\begin{frame}[fragile]{Les types \texttt{integer} et \texttt{numeric}}
\protect\hypertarget{les-types-integer-et-numeric}{}
\begin{itemize}
\tightlist
\item
  Numérique, virgule flottante (\texttt{double} ou \texttt{numeric}),
  base des calculs statistiques
\end{itemize}

\tiny

\begin{Shaded}
\begin{Highlighting}[]
\DecValTok{1}
\end{Highlighting}
\end{Shaded}

\begin{verbatim}
## [1] 1
\end{verbatim}

\normalsize

\tiny

\begin{Shaded}
\begin{Highlighting}[]
\NormalTok{a }\OtherTok{\textless{}{-}} \DecValTok{1} 
\FunctionTok{class}\NormalTok{(a)}
\end{Highlighting}
\end{Shaded}

\begin{verbatim}
## [1] "numeric"
\end{verbatim}

\normalsize

\begin{itemize}
\tightlist
\item
  Numérique, entier (« integer ») : un nombre entier de ce type est
  suivi d'un L.
\end{itemize}

\tiny

\begin{Shaded}
\begin{Highlighting}[]
\NormalTok{b }\OtherTok{\textless{}{-}} \DecValTok{1}\DataTypeTok{L}
\FunctionTok{class}\NormalTok{(b)}
\end{Highlighting}
\end{Shaded}

\begin{verbatim}
## [1] "integer"
\end{verbatim}

\normalsize
\end{frame}

\begin{frame}[fragile]{Le type \texttt{logical} (booleen)}
\protect\hypertarget{le-type-logical-booleen}{}
\begin{itemize}
\tightlist
\item
  Logique/booléen (« logical », résultat des tests)
\end{itemize}

\tiny

\begin{Shaded}
\begin{Highlighting}[]
\NormalTok{a }\SpecialCharTok{==}\NormalTok{ b}
\end{Highlighting}
\end{Shaded}

\begin{verbatim}
## [1] TRUE
\end{verbatim}

\normalsize

\tiny

\begin{Shaded}
\begin{Highlighting}[]
\NormalTok{c }\OtherTok{\textless{}{-}} \ConstantTok{FALSE}
\FunctionTok{class}\NormalTok{(c)}
\end{Highlighting}
\end{Shaded}

\begin{verbatim}
## [1] "logical"
\end{verbatim}

\normalsize
\end{frame}

\begin{frame}[fragile]{Le type \texttt{character}}
\protect\hypertarget{le-type-character}{}
\begin{itemize}
\tightlist
\item
  Caractère (« character », chaînes de contenu quelconque, entre simples
  ou doubles quotes
\end{itemize}

\tiny

\begin{Shaded}
\begin{Highlighting}[]
\NormalTok{d }\OtherTok{\textless{}{-}} \StringTok{\textquotesingle{}Je suis une chaîne de caractères\textquotesingle{}}
\FunctionTok{class}\NormalTok{(d)}
\end{Highlighting}
\end{Shaded}

\begin{verbatim}
## [1] "character"
\end{verbatim}

\normalsize
\end{frame}

\begin{frame}[fragile]{Les types de données élémentaires: \texttt{raw}}
\protect\hypertarget{les-types-de-donnuxe9es-uxe9luxe9mentaires-raw}{}
\begin{itemize}
\tightlist
\item
  Brut (« raw », pour des manipulations au niveau de l'octet)
\end{itemize}

\tiny

\begin{Shaded}
\begin{Highlighting}[]
\FunctionTok{as.raw}\NormalTok{(}\DecValTok{2}\NormalTok{)}
\end{Highlighting}
\end{Shaded}

\begin{verbatim}
## [1] 02
\end{verbatim}

\normalsize

\begin{itemize}
\tightlist
\item
  La fonction \texttt{charToRaw()} convertit un caractère de longueur 1
  en \texttt{raw}
\end{itemize}

\tiny

\begin{Shaded}
\begin{Highlighting}[]
\FunctionTok{charToRaw}\NormalTok{(}\StringTok{"a"}\NormalTok{)}
\end{Highlighting}
\end{Shaded}

\begin{verbatim}
## [1] 61
\end{verbatim}

\normalsize
\end{frame}

\begin{frame}[fragile]{Les vecteurs}
\protect\hypertarget{les-vecteurs}{}
\begin{itemize}
\tightlist
\item
  Un donnée d'un des types atomiques ne peut exister qu'au sein d'un
  \textbf{vecteur}
\end{itemize}

\tiny

\begin{Shaded}
\begin{Highlighting}[]
\NormalTok{a }\OtherTok{\textless{}{-}} \DecValTok{1}
\FunctionTok{is.vector}\NormalTok{(a)}
\end{Highlighting}
\end{Shaded}

\begin{verbatim}
## [1] TRUE
\end{verbatim}

\normalsize

\begin{itemize}
\tightlist
\item
  L'objet \texttt{a} est un vecteur de longueur 1
\end{itemize}

\tiny

\begin{Shaded}
\begin{Highlighting}[]
\FunctionTok{length}\NormalTok{(a)}
\end{Highlighting}
\end{Shaded}

\begin{verbatim}
## [1] 1
\end{verbatim}

\normalsize

\begin{itemize}
\tightlist
\item
  L'objet \texttt{b} est également de type \texttt{numeric}
\end{itemize}

\tiny

\begin{Shaded}
\begin{Highlighting}[]
\NormalTok{b }\OtherTok{\textless{}{-}} \FunctionTok{c}\NormalTok{(}\DecValTok{1}\NormalTok{,}\DecValTok{2}\NormalTok{)}
\FunctionTok{class}\NormalTok{(b)}
\end{Highlighting}
\end{Shaded}

\begin{verbatim}
## [1] "numeric"
\end{verbatim}

\normalsize

\begin{itemize}
\tightlist
\item
  Sa longueur est de 2
\end{itemize}

\tiny

\begin{Shaded}
\begin{Highlighting}[]
\FunctionTok{length}\NormalTok{(b)}
\end{Highlighting}
\end{Shaded}

\begin{verbatim}
## [1] 2
\end{verbatim}

\normalsize
\end{frame}

\begin{frame}[fragile]{Créer un vecteur (1)}
\protect\hypertarget{cruxe9er-un-vecteur-1}{}
\begin{itemize}
\tightlist
\item
  La plus simple façon de créer un vecteur (de longueur 1) est juste
  d'écrire une constante dans un type de base :
\end{itemize}

\tiny

\begin{Shaded}
\begin{Highlighting}[]
\DecValTok{42}
\end{Highlighting}
\end{Shaded}

\begin{verbatim}
## [1] 42
\end{verbatim}

\normalsize

\begin{itemize}
\tightlist
\item
  Un vecteur peut être vide, par exemple à la suite d'une sélection
  infructueuse. La notation est le type du vecteur suivi de (0) :
\end{itemize}

\tiny

\begin{Shaded}
\begin{Highlighting}[]
\NormalTok{v }\OtherTok{\textless{}{-}} \FunctionTok{logical}\NormalTok{(}\DecValTok{0}\NormalTok{)}
\NormalTok{v}
\end{Highlighting}
\end{Shaded}

\begin{verbatim}
## logical(0)
\end{verbatim}

\begin{Shaded}
\begin{Highlighting}[]
\FunctionTok{length}\NormalTok{(v)}
\end{Highlighting}
\end{Shaded}

\begin{verbatim}
## [1] 0
\end{verbatim}

\normalsize
\end{frame}

\begin{frame}[fragile]{Créer un vecteur (2)}
\protect\hypertarget{cruxe9er-un-vecteur-2}{}
\begin{itemize}
\tightlist
\item
  On peut également utiliser:

  \begin{itemize}
  \tightlist
  \item
    la fonction \texttt{vector()} qui crée un vecteur de type donné et
    de longueur donnée
  \end{itemize}
\end{itemize}

\tiny

\begin{Shaded}
\begin{Highlighting}[]
\FunctionTok{vector}\NormalTok{(}\StringTok{"complex"}\NormalTok{,}\DecValTok{5}\NormalTok{)}
\end{Highlighting}
\end{Shaded}

\begin{verbatim}
## [1] 0+0i 0+0i 0+0i 0+0i 0+0i
\end{verbatim}

\normalsize

\begin{itemize}
\tightlist
\item
  La fonction de collecte \texttt{c()} qui cherche à combiner ses
  arguments en vecteur (lorsque c'est possible : quand ils peuvent être
  mis au même type, sinon elle produit une liste)
\end{itemize}

\tiny

\begin{Shaded}
\begin{Highlighting}[]
\FunctionTok{c}\NormalTok{(}\DecValTok{6}\NormalTok{,}\DecValTok{7}\NormalTok{)}
\end{Highlighting}
\end{Shaded}

\begin{verbatim}
## [1] 6 7
\end{verbatim}

\normalsize

\begin{itemize}
\tightlist
\item
  de nombreuses autres fonctions : la répétition
\end{itemize}

\tiny

\begin{Shaded}
\begin{Highlighting}[]
\FunctionTok{rep}\NormalTok{(}\ConstantTok{TRUE}\NormalTok{,}\DecValTok{5}\NormalTok{)}
\end{Highlighting}
\end{Shaded}

\begin{verbatim}
## [1] TRUE TRUE TRUE TRUE TRUE
\end{verbatim}

\normalsize

\begin{itemize}
\tightlist
\item
  la fabrication de suite d'entiers consécutifs, etc\ldots{}
\end{itemize}

\tiny

\begin{Shaded}
\begin{Highlighting}[]
\DecValTok{1}\SpecialCharTok{:}\DecValTok{10}
\end{Highlighting}
\end{Shaded}

\begin{verbatim}
##  [1]  1  2  3  4  5  6  7  8  9 10
\end{verbatim}

\normalsize
\end{frame}

\begin{frame}[fragile]{Exercice}
\protect\hypertarget{exercice}{}
\begin{itemize}
\tightlist
\item
  Avec la fonction \texttt{as.Date()} convertir la chaîne de caractères
  ``2017-01-01'' en donnée de type « date » -Lui ajouter 1. On passe au
  2 janvier 2017. -Lui ajouter la suite des nombres de 1 à 7.
\item
  Appliquer la fonction \texttt{weekdays()} au résultat que l'on
  mémorisera sous le nom \texttt{jour}. Quel jour de la semaine était le
  1er janvier 2017 ?
\end{itemize}
\end{frame}

\begin{frame}[fragile]{Accéder aux éléments d'un vecteur (1)}
\protect\hypertarget{accuxe9der-aux-uxe9luxe9ments-dun-vecteur-1}{}
\begin{itemize}
\tightlist
\item
  La sélection d'une partie d'un vecteur se fait avec l'opérateur
  ``crochet'' \texttt{{[}...{]}} (c'est à dire une fonction,
  cf.~principe numéro 4) au sein duquel on peut préciser :

  \begin{itemize}
  \tightlist
  \item
    un vecteur de nombres entiers : pour extraire les éléments de
    numéros cité
  \end{itemize}
\end{itemize}

\tiny

\begin{Shaded}
\begin{Highlighting}[]
\NormalTok{(v }\OtherTok{\textless{}{-}} \DecValTok{101}\SpecialCharTok{:}\DecValTok{110}\NormalTok{)}
\end{Highlighting}
\end{Shaded}

\begin{verbatim}
##  [1] 101 102 103 104 105 106 107 108 109 110
\end{verbatim}

\normalsize

\begin{itemize}
\tightlist
\item
  par exemple les éléments 3 à 5
\end{itemize}

\tiny

\begin{Shaded}
\begin{Highlighting}[]
\NormalTok{v[}\DecValTok{3}\SpecialCharTok{:}\DecValTok{5}\NormalTok{]}
\end{Highlighting}
\end{Shaded}

\begin{verbatim}
## [1] 103 104 105
\end{verbatim}

\normalsize

\begin{itemize}
\tightlist
\item
  ou les éléments 5 à 3
\end{itemize}

\tiny

\begin{Shaded}
\begin{Highlighting}[]
\NormalTok{v[}\DecValTok{5}\SpecialCharTok{:}\DecValTok{3}\NormalTok{]}
\end{Highlighting}
\end{Shaded}

\begin{verbatim}
## [1] 105 104 103
\end{verbatim}

\normalsize

\begin{itemize}
\tightlist
\item
  un vecteur de nombres entiers négatifs : pour extraire tous les
  éléments sauf ceux de numéro cités
\end{itemize}

\tiny

\begin{Shaded}
\begin{Highlighting}[]
\NormalTok{v[}\SpecialCharTok{{-}}\DecValTok{3}\SpecialCharTok{:{-}}\DecValTok{5}\NormalTok{]}
\end{Highlighting}
\end{Shaded}

\begin{verbatim}
## [1] 101 102 106 107 108 109 110
\end{verbatim}

\normalsize
\end{frame}

\begin{frame}[fragile]{Accéder aux éléments d'un vecteur (2)}
\protect\hypertarget{accuxe9der-aux-uxe9luxe9ments-dun-vecteur-2}{}
\begin{itemize}
\tightlist
\item
  un vecteur de booléens pour spécifier quels éléments doivent être
  conservés (TRUE) ou non (FALSE)
\end{itemize}

\tiny

\begin{Shaded}
\begin{Highlighting}[]
\NormalTok{v[}\FunctionTok{c}\NormalTok{(}\ConstantTok{TRUE}\NormalTok{,}\FunctionTok{rep}\NormalTok{(}\ConstantTok{FALSE}\NormalTok{,}\DecValTok{8}\NormalTok{),}\ConstantTok{TRUE}\NormalTok{)]}
\end{Highlighting}
\end{Shaded}

\begin{verbatim}
## [1] 101 110
\end{verbatim}

\normalsize

\begin{itemize}
\tightlist
\item
  Le crochet ne fait pas de l'indexation, c'est un véritable opérateur
  entre deux vecteurs.
\item
  L'intérêt majeur de la sélection par vecteur de booléens est que ce
  vecteur peut provenir d'un calcul logique, où on exprime une condition
  sur les éléments. Cette condition peut utiliser n'importe quel élément
  connu de R, et en particulier le vecteur lui-même.
\end{itemize}
\end{frame}

\begin{frame}[fragile]{Accéder aux éléments d'un vecteur (3)}
\protect\hypertarget{accuxe9der-aux-uxe9luxe9ments-dun-vecteur-3}{}
\begin{itemize}
\tightlist
\item
  Exemple : une sélection des éléments dont la parité (pair/impair)
  dépend du positionnement d'un paramètre de nom PARITE (\%\% est le
  reste de la division entière)
\end{itemize}

\tiny

\begin{Shaded}
\begin{Highlighting}[]
\NormalTok{PARITE }\OtherTok{\textless{}{-}} \DecValTok{1}
\NormalTok{v[(v }\SpecialCharTok{\%\%} \DecValTok{2}\NormalTok{)}\SpecialCharTok{==}\NormalTok{PARITE]}
\end{Highlighting}
\end{Shaded}

\begin{verbatim}
## [1] 101 103 105 107 109
\end{verbatim}

\normalsize

\begin{itemize}
\tightlist
\item
  L'opérateur \texttt{\%\%} est appliqué à tous les éléments de
  \texttt{x}
\end{itemize}

\tiny

\begin{Shaded}
\begin{Highlighting}[]
\NormalTok{v }\SpecialCharTok{\%\%} \DecValTok{2}
\end{Highlighting}
\end{Shaded}

\begin{verbatim}
##  [1] 1 0 1 0 1 0 1 0 1 0
\end{verbatim}

\normalsize

\begin{itemize}
\tightlist
\item
  Le test renvoie un vecteur de booléens
\end{itemize}

\tiny

\begin{Shaded}
\begin{Highlighting}[]
\NormalTok{(v }\SpecialCharTok{\%\%} \DecValTok{2}\NormalTok{)}\SpecialCharTok{==}\NormalTok{PARITE}
\end{Highlighting}
\end{Shaded}

\begin{verbatim}
##  [1]  TRUE FALSE  TRUE FALSE  TRUE FALSE  TRUE FALSE  TRUE FALSE
\end{verbatim}

\normalsize
\end{frame}

\begin{frame}[fragile]{Utilisation de \texttt{all()} et \texttt{any()}}
\protect\hypertarget{utilisation-de-all-et-any}{}
\begin{itemize}
\tightlist
\item
  Création d'un vecteur de nombres pairs
\end{itemize}

\tiny

\begin{Shaded}
\begin{Highlighting}[]
\NormalTok{x }\OtherTok{\textless{}{-}} \FunctionTok{seq}\NormalTok{(}\DecValTok{2}\NormalTok{,}\DecValTok{12}\NormalTok{,}\DecValTok{2}\NormalTok{)}
\NormalTok{x }
\end{Highlighting}
\end{Shaded}

\begin{verbatim}
## [1]  2  4  6  8 10 12
\end{verbatim}

\normalsize

\begin{itemize}
\tightlist
\item
  Pour vérifier que tous les nombres sont pairs, on peut utiliser
  \texttt{all()}
\end{itemize}

\tiny

\begin{Shaded}
\begin{Highlighting}[]
\FunctionTok{all}\NormalTok{(x }\SpecialCharTok{\%\%} \DecValTok{2} \SpecialCharTok{==} \DecValTok{0}\NormalTok{)}
\end{Highlighting}
\end{Shaded}

\begin{verbatim}
## [1] TRUE
\end{verbatim}

\normalsize
\end{frame}

\begin{frame}[fragile]{Utilisation de \texttt{all()} et \texttt{any()}}
\protect\hypertarget{utilisation-de-all-et-any-1}{}
\begin{itemize}
\tightlist
\item
  Pour tester si un des éléments de \texttt{x} est impair
\end{itemize}

\tiny

\begin{Shaded}
\begin{Highlighting}[]
\FunctionTok{any}\NormalTok{(x }\SpecialCharTok{\%\%} \DecValTok{2} \SpecialCharTok{!=} \DecValTok{0}\NormalTok{)}
\end{Highlighting}
\end{Shaded}

\begin{verbatim}
## [1] FALSE
\end{verbatim}

\normalsize

\begin{itemize}
\tightlist
\item
  On peut sommer les valeurs booléennes pour obtenir les nombre
  d'éléments concernés
\end{itemize}

\tiny

\begin{Shaded}
\begin{Highlighting}[]
\FunctionTok{sum}\NormalTok{( x }\SpecialCharTok{\textgreater{}} \DecValTok{6}\NormalTok{)}
\end{Highlighting}
\end{Shaded}

\begin{verbatim}
## [1] 3
\end{verbatim}

\normalsize
\end{frame}

\begin{frame}[fragile]{Le crochet n'est qu'un opérateur}
\protect\hypertarget{le-crochet-nest-quun-opuxe9rateur}{}
\begin{itemize}
\tightlist
\item
  Puisque le crochet n'est qu'un opérateur, rien n'interdit d'utiliser
  autre chose qu'un symbole du coté gauche de l'opérateur, en
  particulier un résultat d'un appel de fonction
\end{itemize}

\tiny

\begin{Shaded}
\begin{Highlighting}[]
\FunctionTok{seq}\NormalTok{(}\DecValTok{1}\NormalTok{,}\DecValTok{50}\NormalTok{,}\DecValTok{5}\NormalTok{)[}\DecValTok{1}\SpecialCharTok{:}\DecValTok{5}\NormalTok{]}
\end{Highlighting}
\end{Shaded}

\begin{verbatim}
## [1]  1  6 11 16 21
\end{verbatim}

\normalsize
\end{frame}

\begin{frame}[fragile]{Exercice - Extraire une partie d'un vecteur}
\protect\hypertarget{exercice---extraire-une-partie-dun-vecteur}{}
\begin{itemize}
\tightlist
\item
  En utilisant le vecteur des 7 jours à partir d'aujourd'hui compris
  (fonction \texttt{as.Date()} ou \texttt{Sys.Date()}), quels sont les
  dates dont le jour de la semaine se termine par « di » ?
\end{itemize}

\tiny

\begin{verbatim}
## [1] "lundi"    "mardi"    "mercredi" "jeudi"    "vendredi" "samedi"   "dimanche"
\end{verbatim}

\normalsize

\begin{itemize}
\tightlist
\item
  On pourra utiliser la fonction \texttt{endsWith} ou la fonction
  \texttt{str\_detect} du package \texttt{stringr} avec une expression
  régulière
\end{itemize}
\end{frame}

\begin{frame}[fragile]{La fonction \texttt{which}}
\protect\hypertarget{la-fonction-which}{}
\begin{itemize}
\tightlist
\item
  La fonction which permet de faire un pont entre une sélection par
  booléens ou une sélection par indice : elle restitue les positions des
  éléments vérifiant une condition.
\end{itemize}

\tiny

\begin{Shaded}
\begin{Highlighting}[]
\NormalTok{LETTERS}
\end{Highlighting}
\end{Shaded}

\begin{verbatim}
##  [1] "A" "B" "C" "D" "E" "F" "G" "H" "I" "J" "K" "L" "M" "N" "O" "P" "Q" "R" "S"
## [20] "T" "U" "V" "W" "X" "Y" "Z"
\end{verbatim}

\normalsize - Position des lettres N à P

\tiny

\begin{Shaded}
\begin{Highlighting}[]
\FunctionTok{which}\NormalTok{(LETTERS}\SpecialCharTok{\textgreater{}=}\StringTok{"N"} \SpecialCharTok{\&}\NormalTok{ LETTERS}\SpecialCharTok{\textless{}=}\StringTok{"P"}\NormalTok{)}
\end{Highlighting}
\end{Shaded}

\begin{verbatim}
## [1] 14 15 16
\end{verbatim}

\normalsize - Version plus simple sans `which'

\tiny

\begin{Shaded}
\begin{Highlighting}[]
\FunctionTok{length}\NormalTok{(LETTERS[}\FunctionTok{which}\NormalTok{(LETTERS}\SpecialCharTok{\textgreater{}=}\StringTok{"Z"}\NormalTok{)])}
\end{Highlighting}
\end{Shaded}

\begin{verbatim}
## [1] 1
\end{verbatim}

\normalsize - \textbf{Attention} : which est souvent utilisée dans des
contextes où on n'en a pas besoin grâce à la possibilité de sélection
par booléens
\end{frame}

\begin{frame}[fragile]{Le recyclage des éléments dans les opérations
vectorielles (1)}
\protect\hypertarget{le-recyclage-des-uxe9luxe9ments-dans-les-opuxe9rations-vectorielles-1}{}
\begin{itemize}
\tightlist
\item
  En R, les opérateurs arithmétiques et logiques ainsi qu'un grand
  nombre de fonctions dont \texttt{{[}} sont \textbf{vectorisés}
\item
  Lorsqu'on doit faire quelque chose sur tous les éléments d'un vecteur,
  il n'y a pas à penser à répéter la chose sur chaque élément. C'est R
  qui va se charger de la ``boucle''
\item
  Et, avec les architectures modernes (multi-cores, GPUs), il est même
  possible que les opérations se fassent de façon simultanée !
\end{itemize}
\end{frame}

\begin{frame}[fragile]{Le recyclage des éléments dans les opérations
vectorielles (2)}
\protect\hypertarget{le-recyclage-des-uxe9luxe9ments-dans-les-opuxe9rations-vectorielles-2}{}
\begin{itemize}
\tightlist
\item
  Avec des vecteurs de même longueur
\end{itemize}

\tiny

\begin{Shaded}
\begin{Highlighting}[]
\DecValTok{1}\SpecialCharTok{:}\DecValTok{10} \SpecialCharTok{+} \FunctionTok{rep}\NormalTok{(}\DecValTok{100}\NormalTok{,}\DecValTok{10}\NormalTok{)}
\end{Highlighting}
\end{Shaded}

\begin{verbatim}
##  [1] 101 102 103 104 105 106 107 108 109 110
\end{verbatim}

\normalsize

\begin{itemize}
\tightlist
\item
  Avec deux vecteurs de longueurs différentes, par exemple un calcul
  impliquant un scalaire et un vecteur
\end{itemize}

\tiny

\begin{Shaded}
\begin{Highlighting}[]
\DecValTok{1}\SpecialCharTok{:}\DecValTok{10} \SpecialCharTok{+} \DecValTok{100}
\end{Highlighting}
\end{Shaded}

\begin{verbatim}
##  [1] 101 102 103 104 105 106 107 108 109 110
\end{verbatim}

\normalsize

\begin{itemize}
\tightlist
\item
  R ``recycle'' le vecteur le plus court pour en faire (virtuellement)
  un vecteur de la même longueur que le plus long
\end{itemize}

\tiny

\begin{Shaded}
\begin{Highlighting}[]
\DecValTok{1}\SpecialCharTok{:}\DecValTok{10} \SpecialCharTok{+} \FunctionTok{rep}\NormalTok{(}\DecValTok{100}\NormalTok{,}\DecValTok{11}\NormalTok{)}
\end{Highlighting}
\end{Shaded}

\begin{verbatim}
## Warning in 1:10 + rep(100, 11): la taille d'un objet plus long n'est pas
## multiple de la taille d'un objet plus court
\end{verbatim}

\begin{verbatim}
##  [1] 101 102 103 104 105 106 107 108 109 110 101
\end{verbatim}

\normalsize

\tiny

\begin{Shaded}
\begin{Highlighting}[]
\DecValTok{1}\SpecialCharTok{:}\DecValTok{10} \SpecialCharTok{+} \FunctionTok{c}\NormalTok{(}\DecValTok{100}\NormalTok{,}\DecValTok{200}\NormalTok{)}
\end{Highlighting}
\end{Shaded}

\begin{verbatim}
##  [1] 101 202 103 204 105 206 107 208 109 210
\end{verbatim}

\normalsize
\end{frame}

\begin{frame}[fragile]{Recyclage}
\protect\hypertarget{recyclage}{}
\begin{itemize}
\tightlist
\item
  Le code suivant fonctionne :
\end{itemize}

\tiny

\begin{Shaded}
\begin{Highlighting}[]
\FunctionTok{library}\NormalTok{(dplyr)}
\end{Highlighting}
\end{Shaded}

\begin{verbatim}
## 
## Attachement du package : 'dplyr'
\end{verbatim}

\begin{verbatim}
## Les objets suivants sont masqués depuis 'package:stats':
## 
##     filter, lag
\end{verbatim}

\begin{verbatim}
## Les objets suivants sont masqués depuis 'package:base':
## 
##     intersect, setdiff, setequal, union
\end{verbatim}

\begin{Shaded}
\begin{Highlighting}[]
\NormalTok{mtcars }\SpecialCharTok{\%\textgreater{}\%} \FunctionTok{filter}\NormalTok{(cyl}\SpecialCharTok{==}\StringTok{"6"}\NormalTok{)}
\end{Highlighting}
\end{Shaded}

\begin{verbatim}
## # A tibble: 7 x 11
##     mpg   cyl  disp    hp  drat    wt  qsec    vs    am  gear  carb
##   <dbl> <dbl> <dbl> <dbl> <dbl> <dbl> <dbl> <dbl> <dbl> <dbl> <dbl>
## 1  21       6  160    110  3.9   2.62  16.5     0     1     4     4
## 2  21       6  160    110  3.9   2.88  17.0     0     1     4     4
## 3  21.4     6  258    110  3.08  3.22  19.4     1     0     3     1
## 4  18.1     6  225    105  2.76  3.46  20.2     1     0     3     1
## 5  19.2     6  168.   123  3.92  3.44  18.3     1     0     4     4
## 6  17.8     6  168.   123  3.92  3.44  18.9     1     0     4     4
## 7  19.7     6  145    175  3.62  2.77  15.5     0     1     5     6
\end{verbatim}

\normalsize

\tiny

\begin{Shaded}
\begin{Highlighting}[]
\NormalTok{mtcars }\SpecialCharTok{\%\textgreater{}\%} \FunctionTok{filter}\NormalTok{(cyl}\SpecialCharTok{==}\FunctionTok{c}\NormalTok{(}\StringTok{"6"}\NormalTok{,}\StringTok{"4"}\NormalTok{))}
\end{Highlighting}
\end{Shaded}

\begin{verbatim}
## # A tibble: 8 x 11
##     mpg   cyl  disp    hp  drat    wt  qsec    vs    am  gear  carb
##   <dbl> <dbl> <dbl> <dbl> <dbl> <dbl> <dbl> <dbl> <dbl> <dbl> <dbl>
## 1  21       6 160     110  3.9   2.62  16.5     0     1     4     4
## 2  24.4     4 147.     62  3.69  3.19  20       1     0     4     2
## 3  17.8     6 168.    123  3.92  3.44  18.9     1     0     4     4
## 4  32.4     4  78.7    66  4.08  2.2   19.5     1     1     4     1
## 5  33.9     4  71.1    65  4.22  1.84  19.9     1     1     4     1
## 6  27.3     4  79      66  4.08  1.94  18.9     1     1     4     1
## 7  30.4     4  95.1   113  3.77  1.51  16.9     1     1     5     2
## 8  21.4     4 121     109  4.11  2.78  18.6     1     1     4     2
\end{verbatim}

\normalsize - La bonne solution

\tiny

\begin{Shaded}
\begin{Highlighting}[]
\NormalTok{mtcars }\SpecialCharTok{\%\textgreater{}\%} \FunctionTok{filter}\NormalTok{(cyl }\SpecialCharTok{\%in\%} \FunctionTok{c}\NormalTok{(}\StringTok{"6"}\NormalTok{,}\StringTok{"4"}\NormalTok{))}
\end{Highlighting}
\end{Shaded}

\begin{verbatim}
## # A tibble: 18 x 11
##      mpg   cyl  disp    hp  drat    wt  qsec    vs    am  gear  carb
##    <dbl> <dbl> <dbl> <dbl> <dbl> <dbl> <dbl> <dbl> <dbl> <dbl> <dbl>
##  1  21       6 160     110  3.9   2.62  16.5     0     1     4     4
##  2  21       6 160     110  3.9   2.88  17.0     0     1     4     4
##  3  22.8     4 108      93  3.85  2.32  18.6     1     1     4     1
##  4  21.4     6 258     110  3.08  3.22  19.4     1     0     3     1
##  5  18.1     6 225     105  2.76  3.46  20.2     1     0     3     1
##  6  24.4     4 147.     62  3.69  3.19  20       1     0     4     2
##  7  22.8     4 141.     95  3.92  3.15  22.9     1     0     4     2
##  8  19.2     6 168.    123  3.92  3.44  18.3     1     0     4     4
##  9  17.8     6 168.    123  3.92  3.44  18.9     1     0     4     4
## 10  32.4     4  78.7    66  4.08  2.2   19.5     1     1     4     1
## 11  30.4     4  75.7    52  4.93  1.62  18.5     1     1     4     2
## 12  33.9     4  71.1    65  4.22  1.84  19.9     1     1     4     1
## 13  21.5     4 120.     97  3.7   2.46  20.0     1     0     3     1
## 14  27.3     4  79      66  4.08  1.94  18.9     1     1     4     1
## 15  26       4 120.     91  4.43  2.14  16.7     0     1     5     2
## 16  30.4     4  95.1   113  3.77  1.51  16.9     1     1     5     2
## 17  19.7     6 145     175  3.62  2.77  15.5     0     1     5     6
## 18  21.4     4 121     109  4.11  2.78  18.6     1     1     4     2
\end{verbatim}

\normalsize
\end{frame}

\begin{frame}[fragile]{La fonction \texttt{paste()}}
\protect\hypertarget{la-fonction-paste}{}
\begin{itemize}
\tightlist
\item
  La concaténation de chaînes de caractères se fait avec la fonction
  \texttt{paste0}
\item
  Appliquée à deux vecteurs, elle produit un vecteur de chaînes de
  caractères :
\end{itemize}

\tiny

\begin{Shaded}
\begin{Highlighting}[]
\NormalTok{v }\OtherTok{\textless{}{-}} \FunctionTok{c}\NormalTok{(}\StringTok{"un"}\NormalTok{,}\StringTok{"deux"}\NormalTok{,}\StringTok{"trois"}\NormalTok{,}\StringTok{"quatre"}\NormalTok{,}\StringTok{"cinq"}\NormalTok{)}
\NormalTok{w }\OtherTok{\textless{}{-}} \DecValTok{1}\SpecialCharTok{:}\DecValTok{4}
\FunctionTok{paste0}\NormalTok{(v,w)}
\end{Highlighting}
\end{Shaded}

\begin{verbatim}
## [1] "un1"     "deux2"   "trois3"  "quatre4" "cinq1"
\end{verbatim}

\normalsize - La fonction originale \texttt{paste} introduit un espace:

\tiny

\begin{Shaded}
\begin{Highlighting}[]
\FunctionTok{paste}\NormalTok{(v,w)}
\end{Highlighting}
\end{Shaded}

\begin{verbatim}
## [1] "un 1"     "deux 2"   "trois 3"  "quatre 4" "cinq 1"
\end{verbatim}

\normalsize - L'argument \texttt{collapse} permet de changer le mode de
fonctionnement : la concaténation produit un vecteur d'un seul élément.

\tiny

\begin{Shaded}
\begin{Highlighting}[]
\FunctionTok{paste}\NormalTok{(v, }\AttributeTok{collapse=}\StringTok{";"}\NormalTok{)}
\end{Highlighting}
\end{Shaded}

\begin{verbatim}
## [1] "un;deux;trois;quatre;cinq"
\end{verbatim}

\normalsize
\end{frame}

\begin{frame}[fragile]{Modifier une partie d'un vecteur (1/3)}
\protect\hypertarget{modifier-une-partie-dun-vecteur-13}{}
\begin{itemize}
\tightlist
\item
  Pour modifier une partie d'un vecteur on fera par exemple :
\end{itemize}

\tiny

\begin{Shaded}
\begin{Highlighting}[]
\NormalTok{v }\OtherTok{\textless{}{-}} \DecValTok{101}\SpecialCharTok{:}\DecValTok{110}
\NormalTok{v[}\DecValTok{3}\SpecialCharTok{:}\DecValTok{5}\NormalTok{] }\OtherTok{\textless{}{-}} \DecValTok{0}
\NormalTok{v}
\end{Highlighting}
\end{Shaded}

\begin{verbatim}
##  [1] 101 102   0   0   0 106 107 108 109 110
\end{verbatim}

\normalsize

\begin{itemize}
\tightlist
\item
  Un vecteur est extensible à souhait, R complète les trous :
\end{itemize}

\tiny

\begin{Shaded}
\begin{Highlighting}[]
\NormalTok{v[}\DecValTok{12}\NormalTok{] }\OtherTok{\textless{}{-}} \DecValTok{12}
\NormalTok{v}
\end{Highlighting}
\end{Shaded}

\begin{verbatim}
##  [1] 101 102   0   0   0 106 107 108 109 110  NA  12
\end{verbatim}

\normalsize
\end{frame}

\begin{frame}[fragile]{Modifier une partie d'un vecteur (2/3)}
\protect\hypertarget{modifier-une-partie-dun-vecteur-23}{}
\begin{itemize}
\tightlist
\item
  En R, on ne peut pas modifier un objet !
\item
  Or dans l'écriture \texttt{v{[}3:5{]}\ \textless{}-\ 0}:

  \begin{itemize}
  \tightlist
  \item
    une partie du vecteur semble ``recevoir'' la valeur 0 comme s'il
    s'agissait de cases mémoire dans un langage classique,
  \item
    l'assignation \texttt{\textless{}-} ne semble pas du tout être ici
    l'association d'un nom à une valeur : v{[}3:5{]} n'est pas un
    symbole.
  \end{itemize}
\end{itemize}
\end{frame}

\begin{frame}{Modifier une partie d'un vecteur (3/3)}
\protect\hypertarget{modifier-une-partie-dun-vecteur-33}{}
\begin{itemize}
\tightlist
\item
  En fait cette écriture est un leurre, et sous une apparence
  d'instruction d'affectation de langage de programmation classique,
  elle cache deux choses :

  \begin{itemize}
  \tightlist
  \item
    le recours à une nouvelle fonction, de nom un peu particulier
    {[}\textless-, qui se charge de construire un (nouveau) vecteur
    modifié, et joue en quelque sorte le rôle d'une fonction d'écriture
    de données, alors que la fonction {[} réalise la fonction de
    lecture,
  \item
    un raccourci pour une situation classique. Quand on veut, en R,
    modifier un objet associé à un symbole :

    \begin{enumerate}
    \tightlist
    \item
      on crée une copie modifiée de l'objet indiqué par le symbole,
    \item
      on réassigne le symbole à ce nouvel objet.
    \end{enumerate}
  \end{itemize}
\end{itemize}
\end{frame}

\begin{frame}[fragile]{Supprimer des éléments d'un vecteur}
\protect\hypertarget{supprimer-des-uxe9luxe9ments-dun-vecteur}{}
\begin{itemize}
\tightlist
\item
  En revanche on ne peut pas éliminer un élément d'un vecteur.
\item
  Mais il est possible de sélectionner tous les éléments sauf celui
  qu'on voudrait éliminer, ce qui revient au même si on appelle le
  résultat du même nom que l'original.
\end{itemize}

\tiny

\begin{Shaded}
\begin{Highlighting}[]
\NormalTok{v }\OtherTok{\textless{}{-}}\NormalTok{ v[}\SpecialCharTok{{-}}\DecValTok{2}\NormalTok{]}
\NormalTok{v}
\end{Highlighting}
\end{Shaded}

\begin{verbatim}
##  [1] 101   0   0   0 106 107 108 109 110  NA  12
\end{verbatim}

\normalsize
\end{frame}

\begin{frame}[fragile]{Les matrices (1)}
\protect\hypertarget{les-matrices-1}{}
\begin{itemize}
\tightlist
\item
  Pour les besoins des calculs scientifiques, les matrices existent en R
  sous forme d'une spécialisation de la représentation en vecteur
\item
  Pour créer une matrice on peut commencer par créer un vecteur puis on
  spécifie les dimensions de la matrice en utilisant la forme
  ``réversible'' de la fonction \texttt{dim} : \texttt{dim\textless{}-}
\end{itemize}

\tiny

\begin{Shaded}
\begin{Highlighting}[]
\NormalTok{m }\OtherTok{\textless{}{-}} \FunctionTok{c}\NormalTok{(}\DecValTok{0}\NormalTok{,}\DecValTok{5}\NormalTok{,}\DecValTok{4}\NormalTok{,}\DecValTok{9}\NormalTok{,}\DecValTok{3}\NormalTok{,}\DecValTok{0}\NormalTok{,}\DecValTok{0}\NormalTok{,}\DecValTok{1}\NormalTok{,}\DecValTok{2}\NormalTok{,}\DecValTok{7}\NormalTok{)}
\FunctionTok{dim}\NormalTok{(m) }\OtherTok{\textless{}{-}} \FunctionTok{c}\NormalTok{(}\DecValTok{2}\NormalTok{,}\DecValTok{5}\NormalTok{)}
\NormalTok{m}
\end{Highlighting}
\end{Shaded}

\begin{verbatim}
##      [,1] [,2] [,3] [,4] [,5]
## [1,]    0    4    3    0    2
## [2,]    5    9    0    1    7
\end{verbatim}

\normalsize - Une matrice peut avoir plus de deux dimensions
\end{frame}

\begin{frame}[fragile]{Accès aux éléments d'une matrice}
\protect\hypertarget{accuxe8s-aux-uxe9luxe9ments-dune-matrice}{}
\begin{itemize}
\tightlist
\item
  Les crochets permettent d'accéder à des éléments ou des portions d'une
  matrice qui seront soit des vecteurs soit des matrices :

  \begin{itemize}
  \tightlist
  \item
    un seul élément
  \end{itemize}
\end{itemize}

\tiny

\begin{Shaded}
\begin{Highlighting}[]
\NormalTok{m[}\DecValTok{2}\NormalTok{,}\DecValTok{5}\NormalTok{]}
\end{Highlighting}
\end{Shaded}

\begin{verbatim}
## [1] 7
\end{verbatim}

\normalsize - une ligne

\tiny

\begin{Shaded}
\begin{Highlighting}[]
\NormalTok{m[}\DecValTok{2}\NormalTok{,]}
\end{Highlighting}
\end{Shaded}

\begin{verbatim}
## [1] 5 9 0 1 7
\end{verbatim}

\normalsize - une colonne

\tiny

\begin{Shaded}
\begin{Highlighting}[]
\NormalTok{m[,}\DecValTok{2}\NormalTok{]}
\end{Highlighting}
\end{Shaded}

\begin{verbatim}
## [1] 4 9
\end{verbatim}

\normalsize
\end{frame}

\begin{frame}[fragile]{Sélection des éléments d'une matrice avec des
booléens}
\protect\hypertarget{suxe9lection-des-uxe9luxe9ments-dune-matrice-avec-des-booluxe9ens}{}
\begin{itemize}
\tightlist
\item
  Les opérateurs de comparaison \texttt{\textgreater{}},
  \texttt{\textgreater{}=}, `==', \texttt{!=} s'appliquent à chaque
  élément de la matrice
\end{itemize}

\tiny

\begin{Shaded}
\begin{Highlighting}[]
\NormalTok{m}\SpecialCharTok{\textgreater{}}\DecValTok{4}
\end{Highlighting}
\end{Shaded}

\begin{verbatim}
##       [,1]  [,2]  [,3]  [,4]  [,5]
## [1,] FALSE FALSE FALSE FALSE FALSE
## [2,]  TRUE  TRUE FALSE FALSE  TRUE
\end{verbatim}

\normalsize

\begin{itemize}
\tightlist
\item
  On peut les utiliser pour sélectionner des parties de la matrice
\end{itemize}

\tiny

\begin{Shaded}
\begin{Highlighting}[]
\NormalTok{m[m}\SpecialCharTok{\textgreater{}}\DecValTok{4}\NormalTok{]}
\end{Highlighting}
\end{Shaded}

\begin{verbatim}
## [1] 5 9 7
\end{verbatim}

\normalsize - Le résultat est un vecteur
\end{frame}

\begin{frame}[fragile]{Opérations sur les matrices}
\protect\hypertarget{opuxe9rations-sur-les-matrices}{}
\begin{itemize}
\tightlist
\item
  Opération entre une matrice et une valeur unique
\end{itemize}

\tiny

\begin{Shaded}
\begin{Highlighting}[]
\NormalTok{m }\SpecialCharTok{*} \DecValTok{2}
\end{Highlighting}
\end{Shaded}

\begin{verbatim}
##      [,1] [,2] [,3] [,4] [,5]
## [1,]    0    8    6    0    4
## [2,]   10   18    0    2   14
\end{verbatim}

\normalsize

\begin{itemize}
\tightlist
\item
  Opération entre deux matrices
\end{itemize}

\tiny

\begin{Shaded}
\begin{Highlighting}[]
\NormalTok{m }\SpecialCharTok{{-}}\NormalTok{ m}
\end{Highlighting}
\end{Shaded}

\begin{verbatim}
##      [,1] [,2] [,3] [,4] [,5]
## [1,]    0    0    0    0    0
## [2,]    0    0    0    0    0
\end{verbatim}

\normalsize
\end{frame}

\begin{frame}[fragile]{Les fonctions \texttt{rowSums} et
\texttt{colSums}}
\protect\hypertarget{les-fonctions-rowsums-et-colsums}{}
\begin{itemize}
\tightlist
\item
  Somme des lignes
\end{itemize}

\tiny

\begin{Shaded}
\begin{Highlighting}[]
\FunctionTok{rowSums}\NormalTok{(m)}
\end{Highlighting}
\end{Shaded}

\begin{verbatim}
## [1]  9 22
\end{verbatim}

\normalsize

\begin{itemize}
\tightlist
\item
  Somme des colonnes
\end{itemize}

\tiny

\begin{Shaded}
\begin{Highlighting}[]
\FunctionTok{colSums}\NormalTok{(m)}
\end{Highlighting}
\end{Shaded}

\begin{verbatim}
## [1]  5 13  3  1  9
\end{verbatim}

\normalsize

\begin{itemize}
\tightlist
\item
  Sélection
\end{itemize}

\tiny

\begin{Shaded}
\begin{Highlighting}[]
\NormalTok{m[, }\FunctionTok{colSums}\NormalTok{(m)}\SpecialCharTok{\textgreater{}}\DecValTok{5}\NormalTok{]}
\end{Highlighting}
\end{Shaded}

\begin{verbatim}
##      [,1] [,2]
## [1,]    4    2
## [2,]    9    7
\end{verbatim}

\normalsize
\end{frame}

\begin{frame}[fragile]{Les listes}
\protect\hypertarget{les-listes}{}
\begin{itemize}
\tightlist
\item
  Une liste est une collection d'objets qui peuvent être de types
  \textbf{différents}
\item
  On crée une liste avec la fonction \texttt{list()}. Une liste peut
  être vide : \texttt{list()}.
\item
  Les éléments d'une liste peuvent être nommés. Cela simplifiera l'accès
  ultérieur en précisant des noms et non des positions
\item
  La liste ci-dessous contient des champs des types élémentaires du
  langage R: double, entier, booléen, chaîne de caractères, fonction e
  liste. Trois des champs sont nommés : a, f, l. Les autres ne seront
  accessibles que par leur position.
\end{itemize}

\tiny

\begin{Shaded}
\begin{Highlighting}[]
\NormalTok{liste }\OtherTok{\textless{}{-}} \FunctionTok{list}\NormalTok{(}\AttributeTok{a=}\DecValTok{1}\NormalTok{, }\DecValTok{3}\DataTypeTok{L}\NormalTok{, }\StringTok{"a"}\NormalTok{, }\ConstantTok{TRUE}\NormalTok{, }\AttributeTok{f=}\NormalTok{sum, }\AttributeTok{l=}\FunctionTok{list}\NormalTok{(}\DecValTok{1}\NormalTok{, }\DecValTok{2}\NormalTok{))}
\end{Highlighting}
\end{Shaded}

\normalsize
\end{frame}

\begin{frame}[fragile]{Utilisation des listes en R}
\protect\hypertarget{utilisation-des-listes-en-r}{}
\begin{itemize}
\tightlist
\item
  La liste est la structure de données la plus importante de R car elle
  permet de mémoriser n'importe quelle information complexe.
\item
  Elle est à la source de la définition de nombreux types de données :
  \textbf{data frames}, \textbf{objets graphiques}, \textbf{résultats de
  régression}, \ldots{}
\item
  Ces types de données sont souvent accompagnés d'une redéfinition de la
  fonction d'impression/affichage à l'écran qui cache la structure
  interne en produisant une ``belle'' sortie.
\item
  Pour connaître la structure interne, il faut alors utiliser la
  fonction \texttt{str()} qui affiche un descriptif détaillé.
\end{itemize}

\tiny

\begin{Shaded}
\begin{Highlighting}[]
\FunctionTok{str}\NormalTok{(liste)}
\end{Highlighting}
\end{Shaded}

\begin{verbatim}
## List of 6
##  $ a: num 1
##  $  : int 3
##  $  : chr "a"
##  $  : logi TRUE
##  $ f:function (..., na.rm = FALSE)  
##  $ l:List of 2
##   ..$ : num 1
##   ..$ : num 2
\end{verbatim}

\normalsize
\end{frame}

\begin{frame}[fragile]{Accès aux éléments d'une liste}
\protect\hypertarget{accuxe8s-aux-uxe9luxe9ments-dune-liste}{}
\begin{itemize}
\tightlist
\item
  Deux opérateurs (cf.~principe numéro 4) sont disponibles pour accéder
  à un seul élément d'une liste :

  \begin{itemize}
  \tightlist
  \item
    le double crochet \texttt{{[}{[}...{]}{]}}, avec :

    \begin{itemize}
    \tightlist
    \item
      soit un numéro d'élément (surtout en l'absence de noms)
    \end{itemize}
  \end{itemize}
\end{itemize}

\tiny

\begin{Shaded}
\begin{Highlighting}[]
\NormalTok{liste[[}\DecValTok{5}\NormalTok{]]}
\end{Highlighting}
\end{Shaded}

\begin{verbatim}
## function (..., na.rm = FALSE)  .Primitive("sum")
\end{verbatim}

\normalsize

\begin{verbatim}
- soit un nom d’élément (quand la liste contient des éléments nommés),
\end{verbatim}

\tiny

\begin{Shaded}
\begin{Highlighting}[]
\NormalTok{liste[[}\StringTok{"f"}\NormalTok{]]}
\end{Highlighting}
\end{Shaded}

\begin{verbatim}
## function (..., na.rm = FALSE)  .Primitive("sum")
\end{verbatim}

\normalsize

\begin{itemize}
\tightlist
\item
  le \texttt{\$} qui n'évalue pas son argument de droite (qui doit être
  un nom) et ne permet donc pas de paramétrer l'élément à récupérer
\end{itemize}

\tiny

\begin{Shaded}
\begin{Highlighting}[]
\NormalTok{liste}\SpecialCharTok{$}\NormalTok{f}
\end{Highlighting}
\end{Shaded}

\begin{verbatim}
## function (..., na.rm = FALSE)  .Primitive("sum")
\end{verbatim}

\normalsize
\end{frame}

\begin{frame}[fragile]{Accéder à une partie d'une liste}
\protect\hypertarget{accuxe9der-uxe0-une-partie-dune-liste}{}
\begin{itemize}
\tightlist
\item
  L'opérateur simple crochet permet d'extraire une partie d'une liste
  systématiquement sous forme de liste, même lorsqu'il n'y a qu'un seul
  élément.
\item
  Le fonctionnement est similaire à l'opérateur sur les vecteurs. On a
  le choix entre lui fournir :

  \begin{itemize}
  \tightlist
  \item
    un vecteur de nombres, éventuellement tous négatifs,
  \end{itemize}
\end{itemize}

\tiny

\begin{Shaded}
\begin{Highlighting}[]
\NormalTok{liste[}\DecValTok{5}\NormalTok{]}
\end{Highlighting}
\end{Shaded}

\begin{verbatim}
## $f
## function (..., na.rm = FALSE)  .Primitive("sum")
\end{verbatim}

\normalsize

\begin{itemize}
\tightlist
\item
  un vecteur de booléens,
\item
  un vecteur de noms d'éléments.
\end{itemize}

\tiny

\begin{Shaded}
\begin{Highlighting}[]
\NormalTok{liste[}\SpecialCharTok{{-}}\DecValTok{1}\SpecialCharTok{:{-}}\DecValTok{4}\NormalTok{]}
\end{Highlighting}
\end{Shaded}

\begin{verbatim}
## $f
## function (..., na.rm = FALSE)  .Primitive("sum")
## 
## $l
## $l[[1]]
## [1] 1
## 
## $l[[2]]
## [1] 2
\end{verbatim}

\normalsize
\end{frame}

\begin{frame}[fragile]{« Modifier » une liste}
\protect\hypertarget{modifier-une-liste}{}
\begin{itemize}
\tightlist
\item
  Les opérateurs \texttt{{[}{[}} et \texttt{\$} sont ``réversibles''. On
  peut ainsi positionner un élément dans une liste par sa position ou
  son nom avec \texttt{{[}{[}\textless{}-}, ou par son nom
  \texttt{\$\textless{}-}
\item
  Dans l'exemple ci-dessous, on ajoute un 4 ème élément, le troisième
  est indéfini
\end{itemize}

\tiny

\begin{Shaded}
\begin{Highlighting}[]
\NormalTok{liste }\OtherTok{\textless{}{-}} \FunctionTok{list}\NormalTok{(}\DecValTok{1}\NormalTok{,}\DecValTok{2}\NormalTok{)}
\NormalTok{liste[[}\DecValTok{2}\NormalTok{]]}\OtherTok{\textless{}{-}} \SpecialCharTok{{-}}\DecValTok{1}
\NormalTok{liste[[}\DecValTok{4}\NormalTok{]]}\OtherTok{\textless{}{-}} \SpecialCharTok{{-}}\DecValTok{2}
\NormalTok{liste}
\end{Highlighting}
\end{Shaded}

\begin{verbatim}
## [[1]]
## [1] 1
## 
## [[2]]
## [1] -1
## 
## [[3]]
## NULL
## 
## [[4]]
## [1] -2
\end{verbatim}

\normalsize

\begin{itemize}
\tightlist
\item
  Pour ajouter un élément à unbe liste, celle-ci doit préalablement
  exister
\end{itemize}

\tiny

\begin{Shaded}
\begin{Highlighting}[]
\NormalTok{liste2 }\OtherTok{\textless{}{-}} \FunctionTok{list}\NormalTok{()}
\NormalTok{liste2}\SpecialCharTok{$}\NormalTok{a }\OtherTok{\textless{}{-}} \DecValTok{1}
\end{Highlighting}
\end{Shaded}

\normalsize
\end{frame}

\begin{frame}[fragile]{Convertir une liste en vecteur : \texttt{unlist}
(1)}
\protect\hypertarget{convertir-une-liste-en-vecteur-unlist-1}{}
\begin{itemize}
\tightlist
\item
  Lorsque tous les éléments d'une liste sont de même type, la fonction
  \texttt{unlist} permet de construire un vecteur en ``écrasant'' tous
  les éléments de la liste.
\end{itemize}

\tiny

\begin{Shaded}
\begin{Highlighting}[]
\FunctionTok{list}\NormalTok{(}\DecValTok{1}\NormalTok{, }\DecValTok{2}\NormalTok{, }\DecValTok{3}\NormalTok{) }\SpecialCharTok{\%\textgreater{}\%} \FunctionTok{unlist}\NormalTok{()}
\end{Highlighting}
\end{Shaded}

\begin{verbatim}
## [1] 1 2 3
\end{verbatim}

\normalsize

\begin{itemize}
\tightlist
\item
  Ici les éléments sont convertis en chaîne de caractère
\end{itemize}

\tiny

\begin{Shaded}
\begin{Highlighting}[]
\FunctionTok{list}\NormalTok{(}\DecValTok{1}\NormalTok{, }\StringTok{"a"}\NormalTok{) }\SpecialCharTok{\%\textgreater{}\%} \FunctionTok{unlist}\NormalTok{()}
\end{Highlighting}
\end{Shaded}

\begin{verbatim}
## [1] "1" "a"
\end{verbatim}

\normalsize

\begin{itemize}
\tightlist
\item
  Listes imbriquées
\end{itemize}

\tiny

\begin{Shaded}
\begin{Highlighting}[]
\FunctionTok{list}\NormalTok{(}\DecValTok{1}\NormalTok{, }\FunctionTok{list}\NormalTok{(}\DecValTok{2}\NormalTok{,}\DecValTok{3}\NormalTok{)) }\SpecialCharTok{\%\textgreater{}\%} \FunctionTok{unlist}\NormalTok{()}
\end{Highlighting}
\end{Shaded}

\begin{verbatim}
## [1] 1 2 3
\end{verbatim}

\normalsize
\end{frame}

\begin{frame}[fragile]{Convertir une liste en vecteur : \texttt{unlist}
(2)}
\protect\hypertarget{convertir-une-liste-en-vecteur-unlist-2}{}
\begin{itemize}
\tightlist
\item
  Lorsque les éléments de la liste sont nommés, les éléments du vecteur
  résultat sont - par défaut (voir paramètre ad'hoc) - également nommés
  (comme pour les éléments d'une liste, les élements d'un vecteur
  peuvent être nommés)
\end{itemize}

\tiny

\begin{Shaded}
\begin{Highlighting}[]
\FunctionTok{list}\NormalTok{(}\AttributeTok{a=}\DecValTok{1}\NormalTok{, }\AttributeTok{b=}\DecValTok{2}\NormalTok{, }\AttributeTok{c=}\DecValTok{3}\NormalTok{) }\SpecialCharTok{\%\textgreater{}\%} \FunctionTok{unlist}\NormalTok{()}
\end{Highlighting}
\end{Shaded}

\begin{verbatim}
## a b c 
## 1 2 3
\end{verbatim}

\normalsize
\end{frame}

\begin{frame}[fragile]{Stockage des listes en mémoire (1)}
\protect\hypertarget{stockage-des-listes-en-muxe9moire-1}{}
\begin{itemize}
\tightlist
\item
  En mémoire la structure de liste n'est guère différente de celle d'un
  vecteur afin de permettre le même type d'accès aléatoire.
\end{itemize}

\tiny

\begin{Shaded}
\begin{Highlighting}[]
\NormalTok{liste }\OtherTok{\textless{}{-}} \FunctionTok{list}\NormalTok{(}\AttributeTok{a=}\DecValTok{1}\NormalTok{,}\AttributeTok{b=}\DecValTok{2}\NormalTok{)}
\FunctionTok{is.vector}\NormalTok{(liste)}
\end{Highlighting}
\end{Shaded}

\begin{verbatim}
## [1] TRUE
\end{verbatim}

\normalsize

\begin{itemize}
\tightlist
\item
  Utilisation de \texttt{inspect} pour afficher les attributs internes
\end{itemize}

\tiny

\begin{Shaded}
\begin{Highlighting}[]
\FunctionTok{library}\NormalTok{(pryr)}

\FunctionTok{inspect}\NormalTok{(liste)}
\end{Highlighting}
\end{Shaded}

\begin{verbatim}
## <VECSXP 0x564cf60430e8>
##   <REALSXP 0x564cf43484f8>
##   <REALSXP 0x564cf4348530>
## attributes: 
##   <LISTSXP 0x564cf4903428>
##   tag: 
##     <SYMSXP 0x564cebd878d0>
##   car: 
##     <STRSXP 0x564cf6043128>
##       <CHARSXP 0x564cec08c7b0>
##       <CHARSXP 0x564cec31e928>
##   cdr: 
##     NULL
\end{verbatim}

\normalsize
\end{frame}

\begin{frame}{Stockage des listes en mémoire (2)}
\protect\hypertarget{stockage-des-listes-en-muxe9moire-2}{}
\begin{itemize}
\tightlist
\item
  Mais d'autres éléments de R nécessitent une structure de liste : un
  programme est une liste d'appels de fonctions.
\item
  Pour ses besoins internes, R utilise une structure de liste (« pair
  list ») formée de cellules chaînées entre-elle, directement héritée de
  LISP. Chaque cellule a un « tag » spécifiant son contenu, un « car »
  donnant le contenu et « cdr » indiquant la suite de la liste.
\item
  La navigation dans ce type de structure n'est pas possible sans
  recours aux fonctions internes, mais des conversions sont possibles
  avec la fonction as.list.
\end{itemize}
\end{frame}

\hypertarget{importer-des-donnuxe9es-en-r}{%
\section{Importer des données en R}\label{importer-des-donnuxe9es-en-r}}

\begin{frame}[fragile]{Types de données (1)}
\protect\hypertarget{types-de-donnuxe9es-1}{}
\begin{itemize}
\tightlist
\item
  Données tabulaires (lignes x colonnes avec au croisement une donnée «
  élémentaire »)
\item
  \texttt{import} du package \texttt{rio} : l'outil tous terrains en
  fonction du suffixe
\item
  Paramétrage spécifique pour cas particulier : suivre la piste rio, la
  documentation indique le package utilisé (donc préconisé) et ses
  options
\item
  \texttt{read.fwf} pour les formats à largeur de champ fixe
\item
  \texttt{read.csv} pour les fichiers \texttt{.csv} avec un séparateur
  \texttt{,}ou \texttt{;}
\end{itemize}
\end{frame}

\begin{frame}{Types de données (2)}
\protect\hypertarget{types-de-donnuxe9es-2}{}
\begin{itemize}
\tightlist
\item
  Données structurées non tabulaires : un peu de programmation autour de
  packages standard (cf.~rio)
\item
  Classeurs XLSX de plus d'une feuille
\item
  Fichiers XML : packages XML, xml2
\item
  Fichiers texte non structurés : programmer à l'aide des fonctions de
  manipulation de chaînes de caractères (package stringr)
\item
  Fonctions utiles : readLines , read\_file du package readr
\end{itemize}
\end{frame}

\begin{frame}{Types de données (3)}
\protect\hypertarget{types-de-donnuxe9es-3}{}
\begin{itemize}
\tightlist
\item
  Fichiers binaires (autres que images, sons, films) : programmer autour
  du type de données ``raw''
\item
  Fonctions utiles : open, close, readBin
\end{itemize}
\end{frame}

\begin{frame}{Exemple: Prix des carburtants}
\protect\hypertarget{exemple-prix-des-carburtants}{}
\begin{itemize}
\tightlist
\item
  Un fichier réel de mise à disposition d'informations sur les points de
  vente de carburant :

  \begin{itemize}
  \tightlist
  \item
    localisation
  \item
    période d'ouverture
  \item
    services annexes
  \item
    historique des prix
  \end{itemize}
\end{itemize}

\tiny

\normalsize
\end{frame}

\begin{frame}[fragile]{Lecture du fichier}
\protect\hypertarget{lecture-du-fichier}{}
\begin{itemize}
\tightlist
\item
  On lit l'intégralité du fichier avec la fonction \texttt{readLines()}
\end{itemize}

\tiny

\begin{Shaded}
\begin{Highlighting}[]
\NormalTok{con }\OtherTok{\textless{}{-}} \FunctionTok{file}\NormalTok{(}\StringTok{"../data/PrixCarburants\_quotidien\_20231211.xml"}\NormalTok{, }\AttributeTok{encoding =} \StringTok{"latin1"}\NormalTok{)}
\NormalTok{carburants }\OtherTok{\textless{}{-}} \FunctionTok{readLines}\NormalTok{(con)}
\FunctionTok{close}\NormalTok{(con)}
\FunctionTok{unique}\NormalTok{(}\FunctionTok{Encoding}\NormalTok{(carburants))}
\end{Highlighting}
\end{Shaded}

\begin{verbatim}
## [1] "unknown" "UTF-8"
\end{verbatim}

\normalsize - Le résultat est un (très long) vecteur

\tiny

\begin{Shaded}
\begin{Highlighting}[]
\FunctionTok{head}\NormalTok{(carburants)}
\end{Highlighting}
\end{Shaded}

\begin{verbatim}
## [1] "<?xml version=\"1.0\" encoding=\"ISO-8859-1\" standalone=\"yes\"?>"                     
## [2] "<pdv_liste>"                                                                            
## [3] "  <pdv id=\"1000001\" latitude=\"4620100\" longitude=\"519800\" cp=\"01000\" pop=\"R\">"
## [4] "    <adresse>596 AVENUE DE TREVOUX</adresse>"                                           
## [5] "    <ville>SAINT-DENIS-LèS-BOURG</ville>"                                               
## [6] "    <horaires automate-24-24=\"\">"
\end{verbatim}

\normalsize
\end{frame}

\begin{frame}[fragile]{Sélection des lignes}
\protect\hypertarget{suxe9lection-des-lignes}{}
\begin{itemize}
\tightlist
\item
  Avec la fonction str\_detect de stringr, ne conserver que les lignes
  contenant `pdv' (avec un espace à la fin, donc ne contenant pas
  `pdv\_liste').
\item
  str\_detect est une fonction qui accepte une expression régulière mais
  nous n'en avons pas besoin ici.
\end{itemize}

\tiny

\begin{Shaded}
\begin{Highlighting}[]
\FunctionTok{library}\NormalTok{(stringr)}
\NormalTok{carburants }\OtherTok{\textless{}{-}}\NormalTok{ carburants[}\FunctionTok{str\_detect}\NormalTok{(carburants, }\StringTok{"pdv "}\NormalTok{)]}
\FunctionTok{head}\NormalTok{(carburants)}
\end{Highlighting}
\end{Shaded}

\begin{verbatim}
## [1] "  <pdv id=\"1000001\" latitude=\"4620100\" longitude=\"519800\" cp=\"01000\" pop=\"R\">"
## [2] "  <pdv id=\"1000002\" latitude=\"4621842\" longitude=\"522767\" cp=\"01000\" pop=\"R\">"
## [3] "  <pdv id=\"1000004\" latitude=\"4618800\" longitude=\"524500\" cp=\"01000\" pop=\"R\">"
## [4] "  <pdv id=\"1000007\" latitude=\"4622100\" longitude=\"524500\" cp=\"01000\" pop=\"R\">"
## [5] "  <pdv id=\"1000008\" latitude=\"4619900\" longitude=\"524100\" cp=\"01000\" pop=\"R\">"
## [6] "  <pdv id=\"1000009\" latitude=\"4619600\" longitude=\"522900\" cp=\"01000\" pop=\"R\">"
\end{verbatim}

\normalsize - Avec la fonction \texttt{str\_replace\_all}, éliminer :
\texttt{\textless{}pdv} espaces avant, un après), les
\texttt{\textgreater{}} , les double quotes

\tiny

\begin{Shaded}
\begin{Highlighting}[]
\NormalTok{carburants }\OtherTok{\textless{}{-}} \FunctionTok{str\_replace\_all}\NormalTok{(carburants,}\StringTok{"  \textless{}pdv "}\NormalTok{ , }\StringTok{""}\NormalTok{)}
\NormalTok{carburants }\OtherTok{\textless{}{-}} \FunctionTok{str\_replace\_all}\NormalTok{(carburants, }\StringTok{"}\SpecialCharTok{\textbackslash{}\textbackslash{}\textbackslash{}\textbackslash{}}\StringTok{"}\NormalTok{, }\StringTok{""}\NormalTok{)}
\FunctionTok{head}\NormalTok{(carburants)}
\end{Highlighting}
\end{Shaded}

\begin{verbatim}
## [1] "id=\"1000001\" latitude=\"4620100\" longitude=\"519800\" cp=\"01000\" pop=\"R\">"
## [2] "id=\"1000002\" latitude=\"4621842\" longitude=\"522767\" cp=\"01000\" pop=\"R\">"
## [3] "id=\"1000004\" latitude=\"4618800\" longitude=\"524500\" cp=\"01000\" pop=\"R\">"
## [4] "id=\"1000007\" latitude=\"4622100\" longitude=\"524500\" cp=\"01000\" pop=\"R\">"
## [5] "id=\"1000008\" latitude=\"4619900\" longitude=\"524100\" cp=\"01000\" pop=\"R\">"
## [6] "id=\"1000009\" latitude=\"4619600\" longitude=\"522900\" cp=\"01000\" pop=\"R\">"
\end{verbatim}

\normalsize
\end{frame}

\begin{frame}[fragile]{Exercice : Utilisation des expressions régulières
(1)}
\protect\hypertarget{exercice-utilisation-des-expressions-ruxe9guliuxe8res-1}{}
\begin{itemize}
\tightlist
\item
  Les expressions régulières sont un outil de test de chaînes de
  caractères permettant de reconnaître la présence d'un ensemble de
  chaînes possibles grâce à une syntaxe spécifique : un des caractères
  \texttt{\^{}\ \$\ .\ *\ +\ ?\ \textbar{}\ (\ )\ {[}\ {]}\ \{\ \}\ \textbackslash{}}
\item
  La fonction \texttt{str\_detect} du package \texttt{stringr} est une
  des fonctions réalisant ce genre de tests : elle cherche si son
  premier argument contient quelque chose ressemblant à son second
  argument et répond vrai ou faux.
\item
  Le premier argument peut être un vecteur, le résultat sera un vecteur
  de booléens.
\end{itemize}
\end{frame}

\begin{frame}[fragile]{Exercice : Utilisation des expressions régulières
(2)}
\protect\hypertarget{exercice-utilisation-des-expressions-ruxe9guliuxe8res-2}{}
\begin{itemize}
\tightlist
\item
  Les éléments de syntaxe les plus fréquemment utilisés :
\end{itemize}

\tiny

\begin{Shaded}
\begin{Highlighting}[]
\FunctionTok{str\_detect}\NormalTok{(arg,}\StringTok{"\^{}0"}\NormalTok{) }\CommentTok{\# commence par ‘0’}
\FunctionTok{str\_detect}\NormalTok{(arg,}\StringTok{"0$"}\NormalTok{) }\CommentTok{\# se termine par ‘0’}
\FunctionTok{str\_detect}\NormalTok{(arg,}\StringTok{"\^{}}\SpecialCharTok{\textbackslash{}\textbackslash{}}\StringTok{$"}\NormalTok{) }\CommentTok{\# commence par ‘$’ non interprété}

\FunctionTok{str\_detect}\NormalTok{(arg,}\StringTok{"\^{}.0"}\NormalTok{) }\CommentTok{\# commence par n’importe quel caractère puis un 0}
\FunctionTok{str\_detect}\NormalTok{(arg,}\StringTok{"\^{}[1{-}6]"}\NormalTok{) }\CommentTok{\# commence par un caractère de ‘1’ à ‘6’}
\FunctionTok{str\_detect}\NormalTok{(arg,}\StringTok{"\^{}[1{-}68]"}\NormalTok{) }\CommentTok{\# commence par un caractère de ‘1’ à ‘6’ ou un ‘8’}
\FunctionTok{str\_detect}\NormalTok{(arg,}\StringTok{"\^{}[\^{}1{-}6]"}\NormalTok{) }\CommentTok{\# commence par tout sauf un caractère de ‘1’ à ‘6’}
\FunctionTok{str\_detect}\NormalTok{(arg,}\StringTok{"\^{}}\SpecialCharTok{\textbackslash{}\textbackslash{}}\StringTok{d"}\NormalTok{) }\CommentTok{\# commence par un chiffre décimal}

\FunctionTok{str\_detect}\NormalTok{(arg,}\StringTok{"\^{}.*9"}\NormalTok{) }\CommentTok{\# commence par 0 et + caractères quelconques puis un 9}
\FunctionTok{str\_detect}\NormalTok{(arg,}\StringTok{"\^{}.+9"}\NormalTok{) }\CommentTok{\# commence par 1 et + caractères quelconques puis un 9}
\FunctionTok{str\_detect}\NormalTok{(arg,}\StringTok{"\^{}01?"}\NormalTok{) }\CommentTok{\# commence par 0 et éventuellement un 1}
\FunctionTok{str\_detect}\NormalTok{(arg,}\StringTok{"\^{}0(19)?"}\NormalTok{) }\CommentTok{\# commence par 0 et éventuellement le groupe ‘19’}
\FunctionTok{str\_detect}\NormalTok{(arg,}\StringTok{"\^{}(19|20)"}\NormalTok{)}\CommentTok{\# commence par ‘19’ ou ‘20’}
\end{Highlighting}
\end{Shaded}

\normalsize
\end{frame}

\begin{frame}[fragile]{Exercice}
\protect\hypertarget{exercice-1}{}
\begin{itemize}
\tightlist
\item
  Avec la fonction str\_split, éclater la colonne issue du data.frame
  précédent pour séparer en une liste de couples nom=valeur.
\item
  Afficher la structure de la première ligne de la table résultat.
\item
  La table obtenue précédemment n'est pas « propre » (des données non
  élémentaires dans la dernière colonne).
\item
  Avec la fonction \texttt{unnest} de \texttt{tidyr}, répartir la
  nouvelle colonne de type liste sur plusieurs lignes.
\item
  Avec la fonction \texttt{separate}, séparer les deux composants des
  couples nom=valeur dans deux nouvelles colonnes.
\item
  Le résultat peut être considéré comme une table en format long : un
  colonne indique le nom de la donnée, une autre la valeur. Mettre le
  résultat en format « large » (fonction spread ou équivalent): une
  colonne par modalité du nom.
\item
  Avec la fonction \texttt{gf\_point} de \texttt{ggformula}, faire une
  carte des latitudes - longitudes, en les ayant préalablement
  converties en numérique.
\end{itemize}
\end{frame}

\hypertarget{programmation-avec-r}{%
\section{Programmation avec R}\label{programmation-avec-r}}

\begin{frame}{Introduction}
\protect\hypertarget{introduction-1}{}
\begin{itemize}
\tightlist
\item
  Le statique: les données

  \begin{enumerate}
  \tightlist
  \item
    Pas de variables, mais des \textbf{objets} et des \textbf{symboles}
    2, Un objet n'est \textbf{plus modifiable} après sa création
  \item
    Les données des types de base n'existent qu'au sein de
    \textbf{vecteurs}
  \end{enumerate}
\item
  Le dynamique: les fonctions

  \begin{enumerate}
  \setcounter{enumi}{3}
  \tightlist
  \item
    Une fonction derrière toute opération
  \item
    Une fonction est une forme particulière d'objet
  \item
    Toute fonction peut être surchargée
  \item
    Chaque fonction est libre d'interpréter ses arguments comme elle le
    veut
  \end{enumerate}
\end{itemize}
\end{frame}

\begin{frame}{Quelques principes de programmation (1)}
\protect\hypertarget{quelques-principes-de-programmation-1}{}
\begin{itemize}
\tightlist
\item
  Commencez petit: -Pour résoudre un problème `truc', ne pas commencer
  par écrire `truc \textless- function\ldots{}'.

  \begin{itemize}
  \tightlist
  \item
    Mais commencer par décomposer le problème en sous questions
    élémentaires et ensuite commencer par coder et tester ses
    sous-questions.
  \item
    L'assemblage n'est que la dernière étape.
  \end{itemize}
\end{itemize}
\end{frame}

\begin{frame}{Quelques principes de programmation (2)}
\protect\hypertarget{quelques-principes-de-programmation-2}{}
\begin{itemize}
\tightlist
\item
  Écrivez des petites fonctions.

  \begin{itemize}
  \tightlist
  \item
    Une fonction pour chaque opération élémentaire : ne pas tenter de
    faire plusieurs choses disjointes dans une même fonction.
  \item
    Le code d'une fonction doit tenir sur un seul écran pour permettre
    d'en suivre la logique de déroulement sans toucher au clavier et à
    la souris.
  \item
    Les fonctions potentiellement neutres (accolades, return) ne sont
    pas de bons amis quand elle sont sur-utilisées.
  \end{itemize}
\end{itemize}
\end{frame}

\begin{frame}{Quelques principes de programmation (3)}
\protect\hypertarget{quelques-principes-de-programmation-3}{}
\begin{itemize}
\tightlist
\item
  Faites la chasse aux clones.

  \begin{itemize}
  \tightlist
  \item
    Ne JAMAIS dupliquer de code : cela alourdit le programme et
    introduit un point de faiblesse en cas de modification ultérieure.\\
  \end{itemize}
\item
  Factoriser au maximum.

  \begin{itemize}
  \tightlist
  \item
    R permet une grande souplesse dans les arguments des fonctions :des
    codes presque identiques peuvent toujours être réduits à l'usage
    d'une unique fonction.
  \end{itemize}
\end{itemize}
\end{frame}

\begin{frame}{Quelques principes de programmation (4)}
\protect\hypertarget{quelques-principes-de-programmation-4}{}
\begin{itemize}
\item
  Respectez la symétrie. Si le problème à traiter présente une forme de
  symétrie, celle-ci doit se retrouver dans le code, sinon c'est un
  indice de cas mal couverts. -Ne vous préoccupez pas d'optimisation,
  sauf dans les cas critiques -Testez Commencer par les cas extrêmes.
  Les autres ont plus de chances de fonctionner.
\item
  Adoptez un style et tenez vous y.
\item
  Il y a plusieurs façons d'écrire du R, de nommer ses objets ou de
  présenter les programmes.
\item
  Ne mélangez les patois R qu'en cas de nécessité.
\item
  Un nom d'objet parlant évite bien des commentaires.
\item
  Décrivez ce que font vos fonctions.
\item
  Commentez les passages difficiles.
\item
  Mais uniquement les passages difficiles : un commentaire de type
  paraphrase alourdit les programmes et floute la vision.
\item
  Soignez l'esthétique ! Votre programme est votre bébé : gardez le
  propre, bien proportionné et beau. Plus il sera beau plus il sera
  attachant et incitera à lire votre code ou à lui faire confiance.
\end{itemize}
\end{frame}

\hypertarget{duxe9finir-des-fonctions}{%
\section{Définir des fonctions}\label{duxe9finir-des-fonctions}}

\begin{frame}[fragile]{Création d'une fonction}
\protect\hypertarget{cruxe9ation-dune-fonction}{}
\begin{enumerate}
\tightlist
\item
  Lister les noms des paramètres de la fonction
\item
  Ecrire une expression utilisant ces paramètres
\item
  Donner un nom à la fonction
\item
  Exécuter le code donnant la définition pour créer la fonction dans
  l'environnement courant
\end{enumerate}

\tiny

\begin{Shaded}
\begin{Highlighting}[]
\NormalTok{Somme }\OtherTok{\textless{}{-}} \ControlFlowTok{function}\NormalTok{(début,fin)}
  \FunctionTok{sum}\NormalTok{(}\FunctionTok{c}\NormalTok{(}\SpecialCharTok{{-}}\DecValTok{1}\NormalTok{,}\DecValTok{1}\NormalTok{)}\SpecialCharTok{*}\NormalTok{(début}\SpecialCharTok{:}\NormalTok{fin)}\SpecialCharTok{\^{}}\DecValTok{2}\NormalTok{)}
\end{Highlighting}
\end{Shaded}

\normalsize

\begin{enumerate}
\setcounter{enumi}{4}
\tightlist
\item
  Utilisation de la fonction
\end{enumerate}

\tiny

\begin{Shaded}
\begin{Highlighting}[]
\FunctionTok{Somme}\NormalTok{(}\DecValTok{1}\NormalTok{,}\DecValTok{100}\NormalTok{)}
\end{Highlighting}
\end{Shaded}

\begin{verbatim}
## [1] 5050
\end{verbatim}

\normalsize
\end{frame}

\end{document}
